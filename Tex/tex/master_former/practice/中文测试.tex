\documentclass{beamer}
\usetheme{Warsaw}
\usepackage{CJKutf8}
\usepackage{amsmath}
\begin{document}
\begin{CJK}{UTF8}{gkai}
\title{硕士论文框架}
\date{\today}
\author{潘宝祥}
\maketitle
\begin{frame}
\frametitle{Outline}
\begin{itemize}
\item 研究问题阐述
\begin{itemize}
\item 研究现状
\item 研究意义
\item 需要解决的问题
\end{itemize}
\item 研究方法与步骤
\begin{itemize}
\item 建立基本随机方程与水文变量随机过程描述(已完成)
\item 研究意义
\item 需要解决的问题
\end{itemize}
\end{itemize}
\end{frame}


\begin{frame}
只有一件我念念不忘的事,没有改变:我始终是急于要发现,有多少东西我们能说是知道,以及知道的确定性或未定性究竟到什么程度。
\end{frame}


\end{CJK}

\end{document}