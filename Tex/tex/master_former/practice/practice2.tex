\documentclass{beamer}
\usetheme{Warsaw}
\usepackage{amsmath}
\usepackage{CJKutf8}
\begin{CJK}{UTF8}{gbsn}
\begin{document}

\title{Spatial Upscaling Of Stochastic Soil Moisture Model}
\author{Pan Baoxiang}
\date{\today}
\maketitle

\begin{frame}
\frametitle{Research Objectives}
\begin{itemize}
\item Clarify and Generalize the Stochastic Soil Moisture Equation.
\item Derive the random process differential equation of Run-off, EP and leakage.
\item Spatial Upscaling of Stochastic Soil Moisture Model.
\end{itemize}
\end{frame}
\begin{frame}
\frametitle{General Function}
\begin{equation}
\begin{split}
f(s,t+dt)ds=&\underbrace{[1-p_{rain}]f(s+\Delta s,t)d(s+\Delta s)}_{no-rain}\\&+\underbrace{p_{rain} \int_{0}^{s} f(z+\Delta z,t)p_{i|z}(s-z)d(z+\Delta z)ds}_{rain}
\end{split}
\end{equation}
\end{frame}

\begin{frame}
\frametitle{Rainfall Probability}
\begin{equation}
\begin{split}
p_{rain}=&\int_t^{t+dt} \lambda(x)dx\\=&\lambda(t)dt+o(dt)
\end{split}
\end{equation}
\end{frame}

\begin{frame}
\frametitle{No Rain Condition}
For the no-rain condition, the soil moisture loss during time $(t,t+dt)$ is $\Delta s$, which is the function of the soil moisture $s$. It could be denoted as follows:
\begin{equation}
\Delta s=\int_t^{t+dt} \rho[s(t)]dt
\end{equation}
Since $s(t+dt)= s$, take the second order Taylor expansion of $\Delta s$, we have:
\begin{equation}
\Delta s=\rho(s)dt+o(dt)
\end{equation}
Thus:
\begin{equation}
\begin{split}
&f(s+\Delta s,t)d(s+\Delta s)\\=&f(s+\rho(s)dt+o(dt),t)d(s+\rho(s)dt+o(dt))
\\=&[f(s,t)+\frac{\partial{f(s,t)}}{\partial s}\rho(s)dt+o(dt)](1+\frac{d\rho(s)}{ds}dt)ds
\\=&[f(s,t)+\frac{\partial{f(s,t)}}{\partial s}\rho(s)dt+f(s,t)\frac{d\rho(s)}{ds}dt+o(dt)]ds
\end{split}
\end{equation}
\end{frame}
\begin{frame}
\frametitle{Rainfall Condition}
The rainfall condition in equation (1) is a typical convolution form that ensures the soil moisture to be within $(s,s+ds)$ at time $t+dt$. $\Delta z$ denotes the soil moisture loss during time $(t,t+dt)$, and 
$z(t+dt)=z$.  Thus, as derived above:
\begin{equation}
\begin{split}
\Delta z=\int_t^{t+dt} \rho[z(t)]dt=\rho(z)dt+o(dt)
\end{split}
\end{equation} 
so,
\begin{equation}
\begin{split}
 &dt\int_{0}^{s} f(z+\Delta z,t)p_{i|z}(s-z)d(z+\Delta z)\\=&dt\int_{0}^{s} f(z,t)p_{i|z-\rho(z)dt-o(dt)}[s-z+\rho(z)dt+o(dt)]dz\\=&dt\int_{0}^{s} f(z,t)\lbrace p_{i|z}(s-z)-\frac{\partial p_{i|z}(s-z)}{\partial z}[\rho(z)dt+o(dt)]\rbrace dz\\=&dt\int_{0}^{s} f(z,t)p_{i|z}(s-z)dz+o(dt)
 \end{split}
\end{equation}
\end{frame}
\begin{frame}
\frametitle{Combination}
Combine (1),(2),(5),(7), we have:
\begin{equation}
 \begin{split}
 &f(s,t+dt)ds\\=&f(s,t)ds+\rho(s)\frac{\partial{f(s,t)}}{\partial s}dtds+f(s,t)\frac{d\rho(s)}{ds}dtds-\lambda(t)f(s,t)dtds\\&+\lambda(t)dt\int_{0}^{s} f(z,t)p_{i|z}(s-z)dzds+o(dt)
 \end{split}
 \end{equation}
 Subtract $f(s,t)ds$ and divide by $ds$ in both sides of equation (11),$lim(dt)\rightarrow0$, we have:
 \begin{equation}
 \frac{\partial{f(s,t)}}{\partial t}=\frac{\partial{[\rho(s)f(s,t)]}}{\partial s}-\lambda(t)f(s,t)+\lambda(t)\int_{0}^{s} f(z,t)p_{i|z}(s-z)dz
 \end{equation}
 Equation(9) is the general form of soil moisture stochastic differential equation.
\end{frame}
\begin{frame}
\frametitle{Specification}
 To be more specific, we notice that f(s,t) is discontinuous at the point $(0,t)$ and $(1,t)$. A Dirac function is introduced to describe f(s,t):
 \begin{equation}
 f(s,t)=g(s,t)+[\delta(s)+\delta(1-s)](1-F)
 \end{equation} 
 where
 \begin{equation}
 F\equiv\int_{0^+}^{1^-} g(z,t)dz
 \end{equation}
 which denotes the probability that the soil moisture is within $(0^+,1^-)$ at time $t$ . $\delta (x)$ is the Dirac function:
 \begin{equation}
 \delta(x)\equiv
 \begin{cases}
 0&x\neq0;\\\infty&x=0
 \end{cases}
 \end{equation}
 and
 \begin{equation}
 \int_{-\infty}^{\infty} \delta(x)dx=1
 \end{equation}
\end{frame}
\begin{frame}
\frametitle{Specification}
we define:
 \begin{equation}
 \begin{split}
 p_0(t)\equiv&\int_0^{0^+} f(z,t)dz\\=&1-\int_{0^+}^1 f(z,t)dz
 \end{split}
 \end{equation}
 Consider that:
 \begin{equation}
 \begin{split}
 &\int_{0}^{0^+} f(z,t)p_{i|z}(s-z)dz\\=&\int_{0}^{0^+} f(z,t)[p_{i|0}(s)+\frac{\partial p_{i|z}(s-z)}{\partial z}z+o(z)]dz
 \\=&p_{i|0}(s)\int_{0}^{0^+} f(z,t)dz
 \\=&p_{i|0}(s)p_0(t)
 \end{split}
 \end{equation}
\end{frame}
\begin{frame}
\frametitle{Specification}
and
 \begin{equation}
 \begin{split}
 &\int_{0^+}^{s} f(z,t)p_{i|z}(s-z)dz\\=&\int_{0}^{s} g(z,t)p_{i|z}(s-z)dz-\int_{0}^{0^+} g(z,t)p_{i|z}(s-z)dz\\=&\int_{0}^{s} g(z,t)p_{i|z}(s-z)dz
 \end{split}
 \end{equation}
 we have:
 \begin{equation}
 \begin{split}
 \frac{\partial{f(s,t)}}{\partial t}=&\frac{\partial{\rho(s)f(s,t)}}{\partial s}-\lambda(t)f(s,t)+\lambda(t)\int_{0}^{s} f(z,t)p_{i|z}(s-z)dz\\=&\frac{\partial{\rho(s)f(s,t)}}{\partial s}-\lambda(t)f(s,t)+\lambda(t)\int_{0}^{s} g(z,t)p_{i|z}(s-z)\\&+\lambda(t)p_0(t)p_{i|0}(s)
 \end{split}
 \end{equation}
\end{frame}
\begin{frame}
\frametitle{$p_0(t)$}
 For the situation that $s=0$, we construct the integral form forward stochastic function:
 \begin{equation}
 \begin{split}
 &p_0(t+dt)\\=&[1-\lambda(t)dt]p_0(t)+[1-\lambda(t)dt]\int_{0^+}^{\rho^{-1}(\frac{\Delta s_{s=0^+}}{dt})}p(u,t)du+o(dt)\\=&[1-\lambda(t)dt]p_0(t)+\int_0^{0^+} p(u,t)du+o(dt)\\=&[1-\lambda(t)dt]p_0(t)+o(dt)
 \end{split}
 \end{equation} 
 Thus:
 \begin{equation}
 \frac{dp_0(t)}{dt}=-\lambda(t) p_0(t)
 \end{equation}
 \begin{equation}
 p_0(t)=p_0(0)e^{-\lambda(t) t}
 \end{equation}
 \end{frame}
 \begin{frame}
 \frametitle{Combination}
 Combine equation(17) and (20), we have the specific form of soil moisture stochastic differential function:  
 \begin{equation}
 \begin{split}
 \frac{\partial{f(s,t)}}{\partial t}=&\frac{\partial{[\rho(s)f(s,t)]}}{\partial s}-\lambda(t)f(s,t)+\lambda(t)\int_{0}^{s} g(z,t)p_{i|z}(s-z)dz\\&+\lambda(t)p_0(0)e^{-\lambda(t) t}p_{i|0}(s)
 \end{split}
 \end{equation}
 \end{frame}
 \begin{frame}
 \frametitle{Stochastic Analysis of Run-off}
The probability that there to be a normalized run-off between $(r,r+dr)$ during period $(t,t+dt)$, which is denoted as $p(r,t)drdt$, could be expressed as follows:
 \begin{equation}
 p(r,t)drdt=\int_{0}^{1} f(z,t)[f_p(r+1-z,t)drdt]dz
 \end{equation}
where $f_p(x,t)dxdt$ denotes the probability that the there to be normalized rainfall depth within $(x, x+dx)$ during $(t, t+dt)$. 
 \end{frame}
  \begin{frame}
 \frametitle{Stochastic Analysis of Run-off}
According to assumption, the rainfall opportunity $\lambda$ and the normalized single rainfall depth $r$ are independent random variables, thus:
\begin{equation}
f_p(r+1-z,t)drdt=p_{rain}p_{r\_depth}(r+1-z)dr
\end{equation}
where
\begin{equation}
p_{rain}=\lambda(t)dt+o(dt)
\end{equation}
Combine equation(22),(23),(24),erase the higher order term, we have:
 \begin{equation}
 p(r,t)=\lambda(t)\int_{0}^{1} f(z,t)f_{r\_depth}(r+1-z)dz
 \end{equation} 
It could be interpreted as that for a single time step, the probability density for there being a run-off generated of depth $r$ equals to $\lambda(t)\int_{0}^{1} f(z,t)f_{r\_depth}(r+1-z)dz$. 
 \end{frame}
\begin{frame}
\frametitle{Stochastic Analysis of EP and Leakage}
\end{frame}
\begin{frame}
\frametitle{Equilibrium Analysis}
 \begin{equation*}
 \begin{split}
 \underbrace{\frac{\partial{f(s,t)}}{\partial t}}_{0}=&\underbrace{\frac{\partial{[\rho(s)f(s,t)]}}{\partial s}-\lambda(t)f(s,t)+\lambda(t)\int_{0}^{s} g(z,t)p_{i|z}(s-z)dz}_{intergration}\\&+\underbrace{\lambda(t)p_0(0)e^{-\lambda(t) t}p_{i|0}(s)}_{0}
 \end{split}
 \end{equation*}

\end{frame}
\begin{frame}
\frametitle{Two Key Factors of Upscaling}
\begin{itemize}
\item Basic physical process abstraction
\item Time abstraction (in time domain upscaling) or space heterogeneity abstraction (in space domain upscaling)
\end{itemize}
\end{frame}

\begin{frame}
\frametitle{Temporal Upscaling of Stochastic Soil Moisture Model}
\begin{center}
stochastic description + law of large numbers.
\end{center} 
\begin{equation}
\lim_{n\to\infty}P\lbrace\vert\sum_{i=1}^n E_i-n\mu\vert<\epsilon\rbrace=1
\end{equation}
\end{frame}
\begin{frame}
\frametitle{Spatial Upscaling}
Consider the general forward differential equation:
\begin{equation*}
\begin{split}
f(s,t+dt)ds=&\underbrace{[1-p_{rain}]f(s+\Delta s,t)d(s+\Delta s)}_{no-rain}\\&+\underbrace{p_{rain} \int_{0}^{s} f(z+\Delta z,t)p_{i|z}(s-z)d(z+\Delta z)ds}_{rain}
\end{split}
\end{equation*}
We assume:
\begin{itemize}
\item Evapotranspiration and Leakage occur uniformly in the study region.
\item There is a distribution pattern of soil water replenishment function.  
\end{itemize}
\end{frame}
\begin{frame}
\frametitle{Space Heterogeneity Abstraction}
\begin{itemize}
\item According to assumption 1, the spatial upscaling of no-rain condition is a linear process which requires no adaptation.
\item Due to the heterogeneity of the soil, we should parameterize the soil replenishment process, which refers to $p_{i|z}(s-z)$.
\end{itemize}
\end{frame}
\begin{frame}
\frametitle{Determination of $p_{i|z}(s-z)$}
$p_{i|z}(s-z)$ denotes the probability density that the soil replenishment equals to $(s-z)$ on the condition that the priori soil moisture equals to z.
 
In the original point scale stochastic soil moisture model, this probability density equals to the probability density that the normalized rainfall depth equals to $(s-z)$, as long as $s<1$, or not smaller than $1-z$, when $s=1$.  
\end{frame}

\begin{frame}
\frametitle{Determination of $p_{i|z}(s-z)$}
In the spatial upscaling condition, $z$ represents the average soil moisture level of the whole region. Given $z$, the more heterogeneous the region is, the more it is likely to generate run-off with the same rainfall input. 

We adopt the idea of Catchment Storage Capacity Curve from Xinanjiang Model to represent the effects caused by heterogeneity.
\end{frame}
\begin{frame}
\begin{equation}
R=
 \begin{cases}
 P+z-1+[1-\frac{P+a}{1+b}]^{1+b}&{a+P\leq (1+b)};\\P+z-1 &{a+P\geq (1+b)}
 \end{cases}
\end{equation}
where:
\begin{equation}
a=(1+b)[1-(1-z^{\frac{1}{1+b}})]
\end{equation}
\end{frame}
\begin{frame}
\frametitle{Determination of $p_{i|z}(s-z)$}
Given:
\begin{equation}
R=
 \begin{cases}
 P+z-1+[1-\frac{P+a}{1+b}]^{1+b}&{a+P\leq (1+b)};\\P+z-1 &{a+P\geq (1+b)}
 \end{cases}
\end{equation}
we have:
\begin{equation}
I\vert z=
 \begin{cases}
 1-z-[1-\frac{P+a}{1+b}]^{1+b}&{a+P\leq (1+b)};\\1-z &{a+P\geq (1+b)}
 \end{cases}
\end{equation}
thus:
\begin{equation}
p_{i|z}(s-z)=
 \begin{cases}
 p([1-\frac{P+a}{1+b}]^{1+b}=1-s)&{a+P\leq (1+b)};\\p(s=1) &{a+P\geq (1+b)}
 \end{cases}
\end{equation}
\end{frame}
\begin{frame}
\frametitle{Determination of $p_{i|z}(s-z)$}
Since:
\begin{equation}
p(s=1)=0
\end{equation}
we have the simplified form:
\begin{equation}
p_{i|z}(s-z)=p_{rain\_depth} \lbrace(1+b)[(1-z^{\frac{1}{1+b}})-(1-s)^{\frac{1}{1+b}}]\rbrace
\end{equation}
and
\begin{equation}
p_{i|0}(s)=p_{rain\_depth} \lbrace(1+b)[1-(1-s)^{\frac{1}{1+b}}]\rbrace
\end{equation}
thus we reached the spatical upscaled stochastic soil moisture equation:
 \begin{equation}
 \begin{split}
 &\frac{\partial{f(s,t)}}{\partial t}=\frac{\partial{[\rho(s)f(s,t)]}}{\partial s}-\lambda(t)f(s,t)\\&+\lambda(t)\int_{0}^{s} g(z,t)p_{rain\_depth} \lbrace(1+b)[(1-z^{\frac{1}{1+b}})-(1-s)^{\frac{1}{1+b}}]\rbrace dz\\&+\lambda(t)p_0(0)e^{-\lambda(t) t}p_{rain\_depth} \lbrace(1+b)[1-(1-s)^{\frac{1}{1+b}}]\rbrace
 \end{split}
 \end{equation}


\end{frame}
\begin{frame}
\frametitle{Application}
\begin{center}
Strategy
\end{center}
Since the stochastic soil moisture model adopts the same hydrologic procedure model as Xinanjiang Model, we first calibrate the Xinanjiang Model in the study region, then apply the calibrated parameters in the stochastic soil moisture model to study the hydrologic pattern of the region.
\end{frame}

\end{CJK}

\end{document}