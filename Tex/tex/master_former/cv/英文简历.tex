% LaTeX resume using res.cls
\documentclass[margin]{res}
%\usepackage{helvetica} % uses helvetica postscript font (download helvetica.sty)
%\usepackage{newcent}   % uses new century schoolbook postscript font 
\setlength{\textwidth}{5.1in} % set width of text portion
\usepackage{float}
\usepackage{graphicx} 
\usepackage{amsmath}
\usepackage{authblk}
\usepackage{booktabs}
\usepackage{caption}
\usepackage{subfigure}
\begin{document}
\begin{resume}
\section{EDUCATION} {\sl Bachelor of Engineering,} Hydrology and Water Resources \\
                      % \sl will be bold italic in New Century Schoolbook (or
	              % any postscript font) and just slanted in
		      %	Computer Modern (default) font
                School of Water Resources and Hydropower Engineering, Wuhan University\\ 
                \\
                {\sl Master of Engineering,} Hydrology and Water Resources \\
                      % \sl will be bold italic in New Century Schoolbook (or
	              % any postscript font) and just slanted in
		      %	Computer Modern (default) font
                Institute of Hydrology and Water Resources, Tsinghua University, expected July 2015 \\
 
\section{COMPUTER \\ SKILLS} {\sl Languages \& Software:} Lisp, C, Python, \LaTeX\ ; Matlab, Mathematica, R,  Mysql,  ,Git, vim, ArcGIS, Grapher, AutoCAD .\\
                {\sl Operating Systems:} Linux, Windows.\\                             
\section{MATH \\ SKILLS}  Analysis,
 Probability Theory, Statistics,
 Stochastic Process, Linear Algebra, Operations Research, Numerical Analysis, Information Theory, Machine Learning.
            

\section{ACADEMIC\\WORK} {\sl Hydrological Models}   \\
                 \begin{itemize}  \itemsep -2pt %reduce space between items
                 \item Conceptual Precipitation-Runoff Models: TOPMODEL, XINANJIANG Model, Shanbei Model, HyMod.
                \item Coupling of Water-Heat Correlation Models \& Stochastic Soil Moisture Model.
                \item  Parameter Estimation: Gene Algorithm, Particle Swarm Optimizer, SCE-UA. 
                \item Data-Driven Models: Support Vector Machine, Artificial Neutral Network.
                \end{itemize}
 
                {\sl Stochastic Hydrology}  \\
                 \begin{itemize}  \itemsep -2pt %reduce space between items
                 \item Hydrological Time Series Analysis
                 \item Information Theory Applied In Hydrology
                 
                 \end{itemize} 
                 {\sl Hydrometeorology}  \\
                 \begin{itemize}  \itemsep -2pt %reduce space between items
                 \item Boundary Layer Theory
                 \item Comparison of Evapotranspiration Equations 
                 
                 \end{itemize} 
                 {\sl Field Experiments}  \\
                 \begin{itemize}  \itemsep -2pt %reduce space between items
                 \item Soil Characteristic Field Experiments and Hydrometeorology Observation Station Construction in the Upper Hehei Basin  
                 \item Plants and Soil-Water Characteristic Investigation in Talimu Watershed, Xinjiang Province 
                 
                 \end{itemize} 
                 \section{ACADEMIC\\EXPERIENCE}  
 {\it Presentation\\} 
 \begin{itemize}
 \item Oral Presentation in the The 2nd and 3rd CAS/THU Hydrology and Water Resource Symposium
 \item Poster Presentation in the American Geophysical Union 2014 Meeting
 \end{itemize}
  {\it Publication\\} 
  \begin{itemize}
  \item Pan, B. and Cong, Z.  Information Analysis of Watershed Hydrological Patterns across Temporal Scales (to be submitted). 
  \end{itemize}
 

\end{resume}
\end{document}





