
%%% Local Variables:
%%% mode: latex
%%% TeX-master: t
%%% End:

\chapter{以流域水文随机方程为基础的空间升尺度分析}










The general and specific stochastic soil moisture functions are as follows:
\begin{equation}
\frac{\partial{f(s,t)}}{\partial t}=\frac{\partial{[\rho(s)f(s,t)]}}{\partial s}-\lambda(t)f(s,t)+\lambda(t)\int_{0}^{s} f(z,t)p_{i|z}(s-z)dz
\end{equation}
\begin{equation}
 \begin{split}
 \frac{\partial{f(s,t)}}{\partial t}=&\frac{\partial{[\rho(s)f(s,t)]}}{\partial s}-\lambda(t)f(s,t)+\lambda(t)\int_{0}^{s} g(z,t)p_{i|z}(s-z)dz\\&+\lambda(t)p_0(0)e^{-\lambda(t) t}p_{i|0}(s)
 \end{split}
 \end{equation}
Since the mass conservation principle can be viewed as applicable in all the scales in hydrology, we could extent the functions above to a macro spatial situation. 
We assume:


1,Evapotranspiration and Leakage occur uniformly in the study region.

2,There is a distribution pattern of soil water replenishment function.



According to assumption 1, the spatial upscaling of no-rain condition is a linear process which requires no adaptation.
Due to the heterogeneity of the soil, we should parameterize the soil replenishment process, which refers to $p_{i|z}(s-z)$.

$p_{i|z}(s-z)$ denotes the probability density that the soil replenishment equals to $(s-z)$ on the condition that the priori soil moisture equals to z.
 
In the original point scale stochastic soil moisture model, this probability density equals to the probability density that the normalized rainfall depth equals to $(s-z)$, as long as $s<1$, or not smaller than $1-z$, when $s=1$.  

In the spatial upscaling condition, $z$ represents the average soil moisture level of the whole region. Given $z$, the more heterogeneous the region is, the more it is likely to generate run-off with the same rainfall input. 

We adopt the two different schemes to deal with heterogeneity of the soil moisture replenish process, namely the Catchment Storage Capacity Curve from Xinanjiang Model and topographic index method from TOPMODEL.
\section{蓄水容量指数分布的流域水文随机模型}
A general introduction of catchment storage capacity curve. (its assumption, application, etc.)
\begin{equation}
R=
 \begin{cases}
 P+z-1+[1-\frac{P+a}{1+b}]^{1+b}&{a+P\leq (1+b)};\\P+z-1 &{a+P\geq (1+b)}
 \end{cases}
\end{equation}
where:
\begin{equation}
a=(1+b)[1-(1-z^{\frac{1}{1+b}})]
\end{equation}

Given:
\begin{equation}
R=
 \begin{cases}
 P+z-1+[1-\frac{P+a}{1+b}]^{1+b}&{a+P\leq (1+b)};\\P+z-1 &{a+P\geq (1+b)}
 \end{cases}
\end{equation}
we have:
\begin{equation}
I\vert z=
 \begin{cases}
 1-z-[1-\frac{P+a}{1+b}]^{1+b}&{a+P\leq (1+b)};\\1-z &{a+P\geq (1+b)}
 \end{cases}
\end{equation}
thus:
\begin{equation}
p_{i|z}(s-z)=
 \begin{cases}
 p([1-\frac{P+a}{1+b}]^{1+b}=1-s)&{a+P\leq (1+b)};\\p(s=1) &{a+P\geq (1+b)}
 \end{cases}
\end{equation}

Since:
\begin{equation}
p(s=1)=0
\end{equation}
we have the simplified form:
\begin{equation}
p_{i|z}(s-z)=p_{rain\_depth} \lbrace(1+b)[(1-z^{\frac{1}{1+b}})-(1-s)^{\frac{1}{1+b}}]\rbrace
\end{equation}
and
\begin{equation}
p_{i|0}(s)=p_{rain\_depth} \lbrace(1+b)[1-(1-s)^{\frac{1}{1+b}}]\rbrace
\end{equation}
thus we reached the spatical upscaled stochastic soil moisture equation:
 \begin{equation}
 \begin{split}
 &\frac{\partial{f(s,t)}}{\partial t}=\frac{\partial{[\rho(s)f(s,t)]}}{\partial s}-\lambda(t)f(s,t)\\&+\lambda(t)\int_{0}^{s} g(z,t)p_{rain\_depth} \lbrace(1+b)[(1-z^{\frac{1}{1+b}})-(1-s)^{\frac{1}{1+b}}]\rbrace dz\\&+\lambda(t)p_0(0)e^{-\lambda(t) t}p_{rain\_depth} \lbrace(1+b)[1-(1-s)^{\frac{1}{1+b}}]\rbrace
 \end{split}
 \end{equation}
We note that as $b\rightarrow0$, which denotes a homogeneous rainfall replenish condition, equation $(40)$ degenerates to the original function.
\section{以地形指数为基础的流域水文随机模型}
Points with larger watershed area and lower water conductivities are easier to be saturated and generate runoff in a catchment.

Assumptions

1.The hydraulic gradient of subsurface flow is equal to the land-surface slope.
\begin{equation}
q_i=T_itan\beta_i
\end{equation}

2.The actual lateral discharge is proportionwal to the specific watershed area (drainage area per unit length of
contour line).
\begin{equation}
q_i=Ra_i
\end{equation}

3.The conductivity is a negative exponent function of saturated underground water depth.
\begin{equation}
T_i=T_oexp(-z_i/S_{zm})
\end{equation}

Deduction:

During a given infinitesimal time period, with a high resolution DEM, the lateral flow converges to a point immediately. 
there would be water accumulation if the speed of water converging at a point exceeds its conductivity speed.

Under steady state conditions:
\begin{equation}
q_i=T_oexp(-z_i/S_{zm})tan\beta_i=Ra_i
\end{equation}
thus:
\begin{equation}
z_i=-S_{zm}ln\frac{Ra_i}{T_otan\beta_i}
\end{equation}
 
