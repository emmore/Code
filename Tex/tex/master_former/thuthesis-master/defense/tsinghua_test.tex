\documentclass{beamer}
\usepackage{latexsym,amssymb,amsmath,amsbsy,amsopn,amstext,xcolor,multicol}
\usepackage{graphicx,wrapfig,fancybox}
\usepackage{pgf,pgfarrows,pgfnodes,pgfautomata,pgfheaps,pgfshade}
\usepackage{CJKutf8}
\usepackage{multirow}
\usepackage{beamerthemeTsinghua}
\setbeamertemplate{footline}[frame number]{}
\title{基于随机特征的流域水文模型时间尺度研究}
\subtitle{硕士论文答辩}
\author{答辩人:潘宝祥\\指导老师:丛振涛}
\institute{水文与水资源研究所\\ \\清华大学}
\date{\small{2015年6月5日}}
\logo{\includegraphics[height=1.2cm]{logo.png}\vspace{220pt}}

\begin{document}
\begin{CJK}{UTF8}{gkai}
\frame{
\titlepage
}

\newcommand{\nologo}{\setbeamertemplate{logo}{}}
{\nologo
  \section*{目录}
  \frame {
    \frametitle{\secname}
    \tableofcontents
  }

  \AtBeginSubsection[] {
  \frame<handout:0> {
  \frametitle{Outline}
  \tableofcontents[current,currentsubsection]
    }
  }

  \section{研究背景与现状}
  \subsection{研究背景}
  \frame{
  \frametitle{不同时间尺度下的水文过程}
\begin{center}
\begin{table*}
 {
\centering
\begin{tabular}{ccc}
\hline  
 &\textcolor[rgb]{0.455,0.204,0.506}{日尺度}&\textcolor[rgb]{0.455,0.204,0.506}{年际尺度}\\
\hline 
\multirow{2}{*}{\textcolor[rgb]{0.455,0.204,0.506}{本构关系}}&降水-产流过程&\multirow{2}{*}{水热耦合}\\
&蒸散发过程&\\
\\
\multirow{4}{*}{\textcolor[rgb]{0.455,0.204,0.506}{控制因素}}&降水量、降水强度&\multirow{2}{*}{降水总量}\\
&潜在蒸散发&\multirow{2}{*}{潜在蒸散发总量}\\
&前期土壤蓄水量&\multirow{2}{*}{水量能量时间分布}\\
&坡度、土地类型等&\\
\\
\textcolor[rgb]{0.455,0.204,0.506}{精度}&几毫米&几十毫米\\
\\
\multirow{3}{*}{\textcolor[rgb]{0.455,0.204,0.506}{经典模型}}&新安江模型&\multirow{2}{*}{径流系数模型}\\
&GBHM模型&\multirow{2}{*}{Budyko模型}\\
&神经网络&\\
\hline 
\end{tabular}
\end{table*}
}
\end{center}
}


  \frame{
  \frametitle{不同时间尺度下的水文过程}
\begin{center}
\onslide<1->{
\begin{table*}
 {
\centering
\begin{tabular}{cccc}
\hline  
 &\textcolor[rgb]{0.455,0.204,0.506}{日尺度}&\textbf{...}&\textcolor[rgb]{0.455,0.204,0.506}{年际尺度}\\
\hline 
\multirow{2}{*}{\textcolor[rgb]{0.455,0.204,0.506}{本构关系}}&降水-产流过程&\multirow{13}{*}{\textbf{...}}&\multirow{2}{*}{水热耦合}\\
&蒸散发过程&\\
\\
\multirow{4}{*}{\textcolor[rgb]{0.455,0.204,0.506}{控制因素}}&降水量、降水强度&&\multirow{2}{*}{降水总量}\\
&潜在蒸散发&&\multirow{2}{*}{潜在蒸散发总量}\\
&前期土壤蓄水量&&\multirow{2}{*}{水量能量时间分布}\\
&坡度、土地类型等&\\
\\
\textcolor[rgb]{0.455,0.204,0.506}{精度}&几毫米&&几十毫米\\
\\
\multirow{3}{*}{\textcolor[rgb]{0.455,0.204,0.506}{经典模型}}&新安江模型&&\multirow{2}{*}{径流系数模型}\\
&GBHM模型&&\multirow{2}{*}{Budyko模型}\\
&神经网络&&\\
\hline 
\end{tabular}
\end{table*}
}
}
\onslide<2->
{
\begin{enumerate}
\uncover<2->{\item  \textcolor[rgb]{1,0,0}{日尺度水文过程}如何在\textcolor[rgb]{1,0,0}{升尺度}后表现为\textcolor[rgb]{1,0,0}{水热耦合}?}
\uncover<3->{\item 各模型适用的\textcolor[rgb]{1,0,0}{时间尺度范围}是什么?}
\uncover<4->{\item 各模型在不同时间尺度上的\textcolor[rgb]{1,0,0}{精度}如何?}
\end{enumerate}
}
\end{center}
}  
  
  \subsection{研究现状}
 \frame{
 \frametitle{水文模型范式}
 \onslide<1->{
模型组成部分:模型结构(Structure)\emph{,} \emph{输入变量($X_i$),输出变量($X_o$),状态变量($S$),模型参数($Para$)}.
}
\onslide<2->{
 \begin{figure*}
\centering
\includegraphics[width=11cm]{modelparadigm.png}
\caption{}
\end{figure*}
 }
\onslide<3->{
\textbf{不确定度来源:}
\begin{itemize}
    \item \textcolor[rgb]{1,0,0}{观测数据}
        \begin{itemize}
            \item 观测数据\textcolor[rgb]{0.455,0.204,0.506}{不准确}
            \item 观测数据\textcolor[rgb]{0.455,0.204,0.506}{不充分}
        \end{itemize}
    \item \textcolor[rgb]{1,0,0}{概化模型}
\end{itemize}
}
}

 \frame{
 \frametitle{水文模型范式}
 
模型组成部分:模型结构(Structure)\emph{,} \emph{输入变量($X_i$),输出变量($X_o$),状态变量($S$),模型参数($Para$)}.
 
 
 \begin{figure*}
\centering
\includegraphics[width=11cm]{modelparadigm.png}
\caption{}
\end{figure*}
 
\textcolor[rgb]{1,0,0}{不准确}、\textcolor[rgb]{1,0,0}{不充分}的观测数据,通过\textcolor[rgb]{1,0,0}{概化模型}数据处理,能够在多大程度上抓住\textcolor[rgb]{1,0,0}{不同时间尺度}下水文过程的\textcolor[rgb]{1,0,0}{信息}?

}
  
  
  
  
  
  
  
  
  
  \frame{
  \frametitle{概率表示————随机土壤水模型}

 \onslide<1->{
 \begin{center}
 \begin{figure*}
\centering
\includegraphics[width=5cm]{random.png}
\caption{}
\end{figure*}
\end{center}
}
 
 
 \onslide<2->
  {
 \begin{table}[H] 
\resizebox{\textwidth}{!}{
\centering
\begin{tabular}{cccccccc}
\hline 
开发者&空间尺度&	降水&	截留&	产流&	蒸散发&	地下水补给&	稳态概率类型\\
\hline 
Cox \& Isham &\multirow{5}{*}{点尺度}&\multirow{5}{*}{复合泊松过程} &\multirow{5}{*}{固定值}&未考虑&线性&\multirow{2}{*}{未考虑}&伽马分布\\

Milly &  &&&\multirow{4}{*}{蓄满产流}&常值&&指数分布 \\


Rodriguez && &&&两阶段线性&线性&三段分布\\
Laios && &&&多段线性&指数&四段分布\\ 
Porporato & &&&&线性&线性&伽马分布\\
\hline 
\end{tabular}
}
\end{table}
}

}

  
 \subsection{技术路线}
 \frame{
 \frametitle{技术路线}
 \begin{center}
 \begin{figure*}
\centering
\includegraphics[width=10cm]{routine.png}
\caption{}
\end{figure*}
\end{center}
 }
 
 
  
  \section{随机土壤水模型} 
  \subsection{概率形式守恒方程}

  \frame{
 \frametitle{概率形式守恒方程}

   \onslide<1->
   {
  \begin{equation*}
\label{sbalance}
nR_{L}\frac{ds}{dt}=I(s,t)-E(s,t)-L(s,t)
\end{equation*}
}
\onslide<2->
{
\centering{$\Downarrow$}
}
 \onslide<3->
 {
\begin{scriptsize}
\begin{equation*}
\label{basic1}
f(s,t+dt)ds= \underbrace{(1-p_{rain})\Bigg \{ f(s+\Delta s,t)d(s+\Delta s) \Bigg \} }_{no-rain} +\underbrace{p_{rain} \int_{0}^{s} f(z,t)p_{i|z}(s-z+\Delta z)dzds}_{rain}
\end{equation*}
\end{scriptsize}
}

 \onslide<4->
 {
\begin{scriptsize}
\begin{equation*}
 \begin{split}
f(s,t+dt)ds=&(1-p_{rain})\times [f(s,t)+\frac{\partial{f(s,t)\rho(s)}}{\partial s}dt+o(dt)]ds\\
&+p_{rain} \times [\int_{0}^{s} f(z,t)p_{i|z}(s-z)dzds+o(dt)]\\
&+p_{rain} \times \int_{0}^{s} f(z,t)\lbrace \frac{\partial p_{i|z}(x)}{\partial x}[k\rho(z)+(1-k)\rho(s)]-\frac{\partial p_{i|z}(x)}{\partial z}k\rho(z)\rbrace dzdsdt
\end{split}
\end{equation*}
\end{scriptsize}
}
}


  \frame{
 \frametitle{概率形式守恒方程}
假定充分小的时段$dt$内发生降水事件的概率为$\lambda(t) dt$,即:
\begin{equation*}
p_{rain}=\lambda(t) dt
\end{equation*}
消去$ds$,$lim(dt)\rightarrow0$,有:

\begin{equation*}
\label{basic3}
 \frac{\partial{f(s,t)}}{\partial t}=\frac{\partial{[\rho(s)f(s,t)]}}{\partial s}-\lambda(t)f(s,t)+\lambda(t)\int_{0}^{s} f(z,t)p_{i|z}(s-z)dz
 \end{equation*}
上式即为土壤水量平衡方程的基本概率形式.
 
}
 

 \frame{
 \frametitle{$s=0$边界条件}
 \onslide<1->
 {
\begin{scriptsize}
\begin{equation*}
\label{basic00}
\begin{split}
p_0(t+dt)=&\underbrace{(1-p_{rain})[p_0(t)+\int_{0^{+}}^{\rho (0)dt} f(s,t)ds]}_{no-rain} +\underbrace{p_{rain} \int_{0}^{kdt}\int_{0}^{s} f(z,t)p_{i|z}(s-z+\Delta z)dzds}_{rain}
\end{split}
\end{equation*}
\end{scriptsize}
}
\onslide<2->
{
 \begin{equation*}
 p_0(t)=p_0(0)e^{-\lambda(t) t}
 \end{equation*} 
}
}

 \frame{
 \frametitle{$s=1$边界条件}
 \onslide<1->
 {
\begin{scriptsize}
\begin{equation*}
\begin{split}
p_1(t+dt)=&\underbrace{(1-p_{rain})\times 0}_{no-rain}+\underbrace{p_{rain} \int_{1}^{1}\int_{0}^{s} f(z,t)p_{i|z}(s-z+\Delta z)dzds}_{rain}
\end{split}
\end{equation*}
\end{scriptsize}
}
\onslide<2->
{
 \begin{equation*}
p_1(t)=0
 \end{equation*} 
}
}


\frame{
 \frametitle{考虑边界条件的概率形式守恒方程}
 
\begin{scriptsize}
\begin{equation*}
\begin{split}
 \frac{\partial{g(s,t)}}{\partial t}=&\frac{\partial{[\rho(s)g(s,t)]}}{\partial s}-\lambda(t)g(s,t)+\\
 &\lambda(t)\int_{0}^{s} g(z,t)p_{i|z}(s-z)dz+\lambda(t)p_0(0)e^{-\lambda(t) t}p_{i|0}(s)
 \end{split}
\end{equation*}
\end{scriptsize}
 
}











  \subsection{概率形式本构方程}
  \frame{
 \frametitle{概率形式本构方程}
 \onslide<1->{
  \begin{table}[H] 
\resizebox{\textwidth}{!}{
\centering
\begin{tabular}{cccccccc}
\hline 
&点尺度&面尺度\\ 
\hline 
产流方程
&
\begin{equation*}
R=
 \begin{cases}
 0&{P+z\leq 1};\\P+z-1 &{p+z>1}
 \end{cases}
\end{equation*}
&
\begin{equation*}
R=
 \begin{cases}
 p+z-1+[1-\frac{p+a}{1+b}]^{1+b}&{a+p\leq 1+b};\\p+z-1 &{a+p> 1+b}
 \end{cases}
\end{equation*}
\\
\\
产流p.d.f.
&
\begin{equation*}
\label{rpoint}
p_{R|z}(x)=f_P(x+1-z)+\delta(x)\int_{0}^{1-z} f_P(u) du 
\end{equation*}
&
\begin{equation*}
R=
 \begin{cases}
 p+z-1+[1-\frac{p+a}{1+b}]^{1+b}&{a+p\leq 1+b};\\p+z-1 &{a+p> 1+b}
 \end{cases}
\end{equation*}\\
\\
入渗方程
&
\begin{equation*}
I\vert z=
 \begin{cases}
 P&{P+z\leq 1};\\1-z &{P+z>1}
 \end{cases}
\end{equation*}
&
\begin{equation*}
I\vert z=
 \begin{cases}
 1-z-[1-\frac{P+a}{1+b}]^{1+b}&{a+P\leq 1+b};\\1-z &{a+P> 1+b}
 \end{cases}
\end{equation*}\\
\\
入渗p.d.f
&
\begin{equation*}
\label{point}
p_{i|z}(x)=f_P(x)+\delta(x-1+z)\int_{1-z}^{\infty} f_P(u) du 
\end{equation*}
&
\begin{equation*}
\label{xaj}
p_{i|z}(x)=f_P\bigg \{(1+b)\big [(1-z)^{\frac{1}{1+b}}-(1-z-x)^{\frac{1}{1+b}}\big ]\bigg \}+\delta(x-1+z)\int_{(1+b)(1-z)^{\frac{1}{1+b}}}^{\infty} f_P(u) du 
\end{equation*}\\
\hline 
\end{tabular}
}
\end{table}
}
\onslide<2->
{
  \begin{table}[H] 
\centering
\begin{tabular}{cccccccc}
\hline 
线性蒸散&分段线性蒸散\\ 
\hline 
 \begin{equation*}
\label{linearep}
\rho (s)=\frac{EP}{nR_L} \times s
\end{equation*}
&
 \begin{equation*}
\rho (s )=
 \begin{cases}
 \frac{\eta}{s^*} s  &s\leq s^{*}\\ 
 \eta &s^*<s\leq s_1\\
 \eta+k\frac{s-s_1}{1-s_1} &s_1<s\leq 1
 \end{cases}
 \end{equation*}\\
\hline 
\end{tabular}
\end{table}
}
  }
  
  
\frame{
\frametitle{小结}
\begin{scriptsize}
\begin{equation*} 
\label{ssd}
\frac{\partial{g(s,t)}}{\partial t}=\frac{\partial{[\rho(s)g(s,t)]}}{\partial s}-\lambda(t)g(s,t)+\lambda(t)\int_{0}^{s} g(z,t)f_{p}(s-z)dz+\lambda(t)p_0(0)e^{-\lambda(t) t}p_{i|0}(s)
\end{equation*}
\end{scriptsize}
\begin{tiny}
\begin{equation*}\small
\label{ssm}
\frac{\partial{g(s,t)}}{\partial t}=\frac{\partial{[\rho(s)g(s,t)]}}{\partial s}-\lambda(t)g(s,t)+\lambda(t)\int_{0}^{s} g(z,t)f_{p}\{(1+b) [(1-z)^{\frac{1}{1+b}}-(1-s)^{\frac{1}{1+b}} ] \}dz+\lambda(t)p_0(0)e^{-\lambda(t) t}p_{i|0}(s)
\end{equation*}
\end{tiny}
}  
  
  
  \subsection{时间升尺度}
  \frame{
   \frametitle{}
	The enumarate item:
    \begin{enumerate}[<+-| alert@+>]
    \item
	 Item start with number
    \item
     This is item 2
  \end{enumerate}
  }
  
  
  
  \section{基于信息理论的水文模拟不确定度评估}
  \subsection{理论分析}
    \frame{
  \frametitle{信息是什么}
  \begin{itemize}
  \item 信息即为不确定度.
  \item 不确定度是概率的减函数.
  \item 独立事件信息量可加.
  \end{itemize}
  因此,对随机变量$X$,${X=a}$ 事件的信息量为$-logP(X=a)$.
   \begin{equation*}
   Information=Bit+Context
   \end{equation*}
  }  
  
  
  
  
  
  
  \frame{
  \frametitle{先验不确定度与后验不确定度}
  \begin{figure*}
\centering
\includegraphics[width=11cm]{theory.png}
\caption{}
\end{figure*}
  }  
  
 \frame{
  \frametitle{贝叶斯公式的“熵”形式}
 \footnotesize{
\begin{itemize}
\item[(1)] 贝叶斯公式对数变换:
\begin{equation*}
\label{log}
logP(A|B) =logP(A)+log \frac{P (AB)}{P(A)P(B)} 
\end{equation*}
\item[(2)] 两边乘以$-P(A,B)$:
\begin{equation*}
\label{element}
-P(A,B)logP(A|B) =-P(A,B) logP(A)-P(A,B) log \frac{P (AB)}{P(A)P(B)} 
\end{equation*} 
\item[(3)] 在概率空间求和或积分:
\begin{tiny}
\begin{equation*}
\label{element1}
-\sum_{A} \sum_{B} P(A,B)logP(A|B) =-\sum_{A} \sum_{B} P(A,B) logP(A)-\sum_{A} \sum_{B} P(A,B) log \frac{P (AB)}{P(A)P(B)}
\end{equation*} 
\end{tiny}
or
\begin{tiny}
\begin{equation*}
\label{element2}
-\int\int P(A,B)logP(A|B)dAdB =-\int\int P(A,B)logP(A)dAdB -\int \int P(A,B)log \frac{P (AB)}{P(A)P(B)}dAdB
\end{equation*}
\end{tiny}
\end{itemize}

\begin{equation*}
\label{bayesuncertainty}
H(A|B) = H(A)-MI(A,B)
\end{equation*}
\begin{center}
\centering{后验不确定度$=$先验不确定度$-$样本信息}
\end{center}
}
  
  }
  
  \frame{
  \frametitle{样本信息\\ ---奥卡姆剃刀与信息处理不等式}
  
  
  }
  
  \frame{
  \frametitle{随机不确定度\&认知不确定度}
  
  
  }
  
  
  \subsection{技术分析}
  \frame{
  \frametitle{分辨率}
  
  
  }
  \frame{
  \frametitle{高维互信息估算}
  
  
  }
  \frame{
  \frametitle{高维互信息估算}
  
  
  }

  \subsection{实例分析}


  \frame{
  \frametitle{\subsecname}
  Description item:
  \begin{description}%[<+-| alert@+>]
  \item[description] Item starts with description.
  \item[Fermion] This is item 2.
  \end{description}
  }
  \section{总结与展望}
  \frame{
  \frametitle{\secname}
  \onslide<1,2,3>
  {
  论文主要工作
  \begin{description}%[<+-| alert@+>]
  \item[description] Item starts with description.
  \item[Fermion] This is item 2.
  \end{description}
  }
  \onslide<2,3>
  {
   缺点与不足
  \begin{description}%[<+-| alert@+>]
  \item[description] Item starts with description.
  \item[Fermion] This is item 2.
  \end{description}
  }
   \onslide<3>
  {
   展望
  \begin{description}%[<+-| alert@+>]
  \item[description] Item starts with description.
  \item[Fermion] This is item 2.
  \end{description}
  }
  }
\subsection*{Thanks}

\frame{
  %\frametitle{}
  \onslide<1->
  {
  \begin{figure}[H]\centering
  \includegraphics[width=0.9\textwidth]{github.png}
  \end{figure}
  \center{https://github.com/morepenn}
  }
  \onslide<2->
  {
  \center{谢谢!欢迎各位老师同学批评指正}
  }
}
}
\end{CJK}
\end{document}

  
%\frame{
%  %\frametitle{\subsecname~ frame b}
%  \begin{itemize}[<+-| alert@+>]
%  \item
%  item a
%  \end{itemize}
%}
%\begin{figure}
%\includegraphics[height=10cm,width=12cm]{a3.eps}
%\caption{}
%\label{a3}
%\end{figure}
