\documentclass{article}
\usepackage{CJKutf8}
\usepackage{amsmath}
\usepackage{listings}
\begin{document}
\begin{CJK}{UTF8}{gkai}
\title{最优化投篮}
\date{\today}
\author{潘宝祥}
\maketitle
定点投篮是篮球运动员最重要的基本功之一。以往的投篮教学主要关注于建立稳定的投篮姿势和培养投篮肌肉记忆这些经验性内容。而在篮球赛场上,篮筐内径几乎两倍于篮球直径,投篮距离一般小于7.25m(三分线),这种宽松的约束条件滋养了大批奇形怪状的投篮方式。本文尝试运用初等力学和微分知识,确定最优化投篮的三个基本的力学要素(投篮出手高度,出手角度与出手速度),为今后合理指导投篮训练,培养运动员良好的投篮技巧提供初步的理论指导。

投篮不中分两种情形,即
\begin{itemize}
\item 投篮侧向偏差,即“篮子正不正”的问题
\item 投篮正向偏差,即“力道对不对”的问题
\end{itemize}
前一问题可通过投手在投篮训练中反馈调节改善。本文关注后一问题。

假定在距篮筐$l$处出手,投手的控制变量为出手点高度$h$,球出手速度$v$与出手角度$\theta$。优化目标为:
\begin{itemize}
\item 单次投篮最省力
\item 球以最大的
\end{itemize}


\end{CJK} 
\end{document}
