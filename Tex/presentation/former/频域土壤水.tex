\documentclass{beamer}
\usetheme{Warsaw}
\usepackage{CJKutf8}
\usepackage{amsmath}
\usepackage{listings}
\usepackage{amsmath}
\usepackage{booktabs}
\usepackage{authblk} 
\usepackage{graphicx} 
\usepackage{diagbox}
\usepackage{indentfirst}
\usepackage{float}
\usepackage{xcolor}
\begin{document}
\begin{CJK}{UTF8}{gkai}
\title{土壤水含量频域分析}
\date{\today}
\author{潘宝祥}
\maketitle

\begin{frame}
\frametitle{Outline}
\begin{itemize}
\item 背景知识
\item 理论推导
\item 理论解释与应用
\end{itemize}
\end{frame}



\begin{frame}
\frametitle{背景知识}
\begin{itemize}
\item 时域与频域
\item 噪声以及噪声的颜色
\item 降水与土壤水的频域表示
\end{itemize}
\end{frame}

\begin{frame}
\frametitle{时域与频域}
一个信号可以在时域或频域中表示. 虽然两种表示方式等价,但表达式复杂程度不同.
\begin{figure}[H]
\centering
\includegraphics[width=8cm]{fourier.png}
\end{figure}
%\begin{equation*}
%F(\omega)=\int_{-\infty}^{\infty}f(t)e^{-i\omega t}dt
%\end{equation*}
%\begin{equation*}
%f(t)=\frac{1}{2\pi}\int_{-\infty}^{\infty}F(\omega)e^{i\omega t}d\omega
%wai\end{equation*}
 
\end{frame}

\begin{frame}
\frametitle{噪声}
噪声是一个随机过程,而随机过程有其频域表示函数,其形状则决定了噪声的``颜色''(性质).

信号的频率特性:
\begin{itemize}
\item \textcolor{red}{频谱(Spectral)}:时域信号在频域下的表示方式. 以振幅及相位为因变量,频率为自变量的函数,包含振幅频谱与相位频谱.
\item \textcolor{red}{频谱密度(Spectral Density)}: 一个能量信号$f(t)$的频域表达式$F(\omega)$.
\item \textcolor{red}{功率/能量谱密度(Power Spectral Density)}:$\Phi(\omega)=\frac{F(\omega)F^*(\omega)}{2\pi}$
\end{itemize}
噪声根据其功率谱密度函数决定``颜色''.
\end{frame}

\begin{frame}
\frametitle{噪声的颜色}
噪声功率谱密度为频率的幂律函数:
\begin{equation*}
\Phi(\omega)=\frac{F(\omega)F^*(\omega)}{2\pi} \propto \frac{1}{w^\beta}
\end{equation*}
\begin{itemize}
\item \textcolor{red}{白噪声} $\beta=0$
\item 粉噪声 $\beta=1$
\item \textcolor{red}{红噪声} $\beta=2$
\item 蓝噪声 $\beta=-1$
\item 紫噪声 $\beta=-2$
\end{itemize}
 
\end{frame}

\begin{frame}
\frametitle{土壤水含量s的时域频域表示}
\begin{figure}[H]
\centering
\includegraphics[width=5cm]{laurenz.png}
\caption{土壤水含量s时域频域洛伦兹图(Gabriel et al.,2007)}
\end{figure} 
s的时域表示分布均衡,频域表示分布不均衡. 在频域上需要很少的系数来表示该信号.
\end{frame}

\begin{frame}
\frametitle{土壤水含量s频域特征提取}
提取能量最大的0.38\% 傅里叶系数,令剩余系数为0,重建信号精度如下:
\begin{figure}[H]
\centering
\includegraphics[width=5cm]{reconstruct.png}
\caption{频域特征提取土壤水信号(Gabriel et al.,2007)}
\end{figure} 
回归斜率 $=0.97$, 截距 $= 0.024$, 相关系数你 $r^2
=0.96$

\end{frame}

\begin{frame}
\frametitle{降水与土壤水的频域表示}
\begin{figure}[H]
\centering
\includegraphics[width=8cm]{noise.png}
\end{figure} 
\end{frame} 



\begin{frame}
\frametitle{理论推导}
\begin{equation*}
\frac{ds(t)}{dt}+\frac{L(t)}{\eta R_L}=\frac{p(t)}{\eta R_L}
\end{equation*} 
两边同时乘以$e^{-ift}$,由$-\infty$到$\infty$对$t$ 积分,得:
\begin{equation*}
\int_{-\infty}^{\infty}e^{-ift}ds(t)+\frac{\int_{-\infty}^{\infty}e^{-ift}L(t)dt}{\eta R_L}=\frac{\int_{-\infty}^{\infty}e^{-ift}p(t)dt}{\eta R_L}
\end{equation*} 
\end{frame}

\begin{frame}
右边等于
\begin{equation*}
\frac{P(f)}{\eta R_L} 
\end{equation*}
左边第一项等于:
\begin{equation*}
\begin{split}
&s(t)e^{-ift}|_{-\infty}^{\infty}+if\int_{-\infty}^{\infty}e^{-ift}s(t)dt\\=&s(t)e^{-ift}|_{-\infty}^{\infty}+ifS(f)
\end{split}
\end{equation*}
假定为平稳过程,则$s(t)e^{-ift}|_{-\infty}^{\infty}=0$(Priestley,1981).
\begin{equation*}
\begin{split}
&s(t)e^{-ift}|_{-\infty}^{\infty}+if\int_{-\infty}^{\infty}e^{-ift}s(t)dt\\=&s(t)e^{-ift}|_{-\infty}^{\infty}+ifS(f)\\=&ifS(f)
\end{split}
\end{equation*}
\end{frame}

\begin{frame}
假设:
\begin{equation*}
L(t)=ET_{max}\times  s(t)
\end{equation*}
则右边第二项等于:
\begin{equation*}
\frac{ET_{max}}{\eta R_L} S(f)
\end{equation*}
将上述三式带入原方程,移项,得:
\begin{equation*}
S(f)=\frac{P(f)}{if\eta R_L+ET_{max}}
\end{equation*}
\end{frame}

\begin{frame}
\frametitle{土壤水含量功率谱密度}
由
\begin{equation*}
E_s (f)=|S(f)|^2
\end{equation*}
得:
\begin{equation*}
E_s (f)=\frac{|P(f)|^2}{\beta ^2+f^2}
\end{equation*}
其中
\begin{equation*}
\beta=\frac{ET_{max}}{\eta R_L}
\end{equation*}
\end{frame}

\begin{frame}
\frametitle{水文解释}
$f/\beta$决定了土壤水含量为白噪音或红噪音.
f相对$\beta$较小时,s为白噪音;
f相对$\beta$较大时,s为红噪音. 
\begin{figure}[H]
\centering
\includegraphics[width=4cm]{experiment.png}
\end{figure}
对某一流域,若$\beta$较小,在较小的模拟时间尺度上($f$较小),土壤水含量信号s即转化为红噪音,相邻尺度间相关关系显著,土壤水记忆较长,必须采用迭代结构的模型求解;相反,若$\beta$较大,模拟时间尺度$f$必须非常大,s才会转化为红噪音,因此,该类地区土壤水记忆较短,估计Budyko模型会得到更好的应用.
\end{frame}


 

\end{CJK}

\end{document}
