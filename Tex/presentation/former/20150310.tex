\documentclass{beamer}
\usetheme{Warsaw}
\usepackage{CJKutf8}
\usepackage{amsmath}
\usepackage{listings}
\usepackage{amsmath}
\usepackage{booktabs}
\usepackage{authblk} 
\usepackage{graphicx} 
\usepackage{diagbox}
\usepackage{indentfirst}
\usepackage{float}
\begin{document}
\begin{CJK}{UTF8}{gkai}
\title{学期计划}
\date{\today}
\author{潘宝祥}
\maketitle

\begin{frame}
\frametitle{Outline}
\begin{itemize}
\item 毕业论文
\item 小论文
\item 申请/工作
\end{itemize}
\end{frame}



\begin{frame}
\frametitle{毕业论文时程}
\begin{itemize}
\item 4月15日  论文初稿
\item 4月22日  中期检查
\item 5月4日   小论文 
\item 5月6日   完整论文
\item 6月初    答辩
\end{itemize}
\end{frame}

\begin{frame}
\frametitle{毕业论文已完成情况}
\begin{itemize}
\item 随机土壤水模型空间升尺度.
\item 单维离散连续降水信息熵计算(概率密度曲线,knn法)与验证(Huffman编码).
\item 高维水文变量互信息计算方法(knn+SVR).
\item 基于熵和互信息的流域水文模拟不确定性分析(80个流域).
\end{itemize}
\end{frame}

\begin{frame}
\frametitle{毕业论文未完成项目}
\begin{itemize}
\item 基于空间升尺度随机土壤水方程的归因分析.
\item 随机土壤水方程频域分析. 
\item 单维离散连续降水信息熵计算核函数法.
\item 基于熵和互信息的流域水文模拟不确定性分析(频域).
\end{itemize}
\end{frame}

\begin{frame}
\frametitle{小论文}
\begin{itemize}
\item Information Analysis of Catchment Hydrological Patterns across Temporal Scales
\item On the Accuracy of Precipitation Entropy Estimation
\item Attribution Analysis of Long Term Catchment Hydrological Behaviour Based on Stochastic Soil Moisture Model
\end{itemize}
\end{frame}

\begin{frame}
\frametitle{申请}
申请
\begin{itemize}
\item UA
\item Nebraska \& Delft
\item Irvine
\item MSU
\end{itemize}
留学基金委
\begin{itemize}
\item  3月18日 网申
\item  3月30日 院系推荐资料
\item 4月10日 校内公示名单
\item 4月12日 学校提交推荐资料
\item 5月 公布录取名单
\end{itemize}
\end{frame}

\begin{frame}
\frametitle{工作}
\begin{itemize}
\item 36kr
\item DJI
\end{itemize}
\end{frame} 

\end{CJK}

\end{document}
