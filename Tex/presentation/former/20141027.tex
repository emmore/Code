\documentclass{beamer}
\usetheme{Warsaw}
\usepackage{amsmath}
\usepackage{authblk}
\usepackage{booktabs}
\usepackage{float}
\usepackage{listings}
\begin{document}

\title{Water Balance Modelling Evaluation through Mapping the Hydrological Pattern to Information Space\\
\raggedleft Results \& Discussion}
\date{\today}
\author {Pan Baoxiang}
\maketitle 

\begin{frame}
\frametitle{Results}
The MI(Input;Observation) \& MI(Output;Observation) at different 
\begin{itemize}
\item Simulation\_ Scale
\item Previous\_ Input
\item Simulation \_ Time
\end{itemize} 
\centering{4\- d $\to$ 3\- d $\to$2\- d}
\end{frame}


\begin{frame}
\frametitle{Simulation\_{Time}-Mutual Information Slice}
The uniformity of the Simulation\_{Time}-Mutual Information Slice represent a possible constant performance of an ideal model when applied in different atmospheric situations.
\end{frame}

\begin{frame}
\frametitle{Results}
\begin{figure}[htbp]
\centering
\includegraphics[width=10cm]{1.jpg}
\end{figure}
\end{frame}

\begin{frame}
\frametitle{Results}
\begin{figure}[htbp]
\centering
\includegraphics[width=10cm]{2.jpg}
\end{figure}
\end{frame}

\begin{frame}
\frametitle{Results}
\begin{figure}[htbp]
\centering
\includegraphics[width=10cm]{3.jpg}
\end{figure}
\end{frame}

\begin{frame}
\frametitle{Previous\_ Input-Mutual Information Slice}
The first stationary point of the Previous\_ Input-Mutual Information Curve of small simulation scale represent the convergent time. 

That of the larger simulation scale represent how the former hydrological condition effects the water movement.
\\

The previous input value of the first stationary point becomes smaller and smaller as the simulation scale expands, disappears at the scale of 40 at this watershed. This is the time-scale where non-iterative model structure could provide satisfactory results.
\end{frame}

\begin{frame}
\frametitle{Simulation\_ Scale-Mutual Information Slice}
The first stationary point of the Simulation\_ Scale-Mutual Information Curve of no previous input represent point when Budyko water-heat correlation dominates the hydrological circulation.
\end{frame}

\begin{frame}
\frametitle{Simulation\_ Time-Mutual Information Slice}
 
\end{frame}

\end{document}
