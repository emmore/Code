\documentclass{beamer}
\usetheme{Warsaw}
 
\usepackage{amsmath}
\usepackage{listings}
\usepackage{amsmath}
\usepackage{booktabs}
\usepackage{float}
\usepackage{graphicx} 
\usepackage{diagbox}
\begin{document}
 
\title{Stochastic Learning and Its Application In Hydrology}
\author{Pan Baoxiang}
\date{\today}
\maketitle

\begin{frame}
\frametitle{Outline}
\begin{itemize}
\item Introduction to Stochastic Learning
\item Two Cases
\begin{itemize}

\item ICA
\item Support Vector Machine

\end{itemize}
 
\item Discussion
\end{itemize}
\end{frame}

\begin{frame}
\frametitle{Introduction}
\textbf{Definition}:Machine learning is a scientific discipline that deals with the construction and study of algorithms that can learn from data. Such algorithms operate by building a model based on inputs  and using that to make predictions or decisions, rather than following only explicitly programmed instructions.
 
\end{frame}

\begin{frame}
\frametitle{Conditions to Use Stochastic Learning}
\begin{itemize}
\item There is pattern
\item There is not explicit mathematical expression
\item Available Data
\end{itemize}
\end{frame}

\begin{frame}
\frametitle{ICA}
Independent Component Analysis (ICA) is a computational method for separating a multivariate signal into additive subcomponents. This is done by assuming that the subcomponents are non-Gaussian signals and that they are statistically independent from each other. ICA is a special case of blind source separation.

\end{frame}

\begin{frame}
\frametitle{Problem Statement}
Linear Regression of $NDVI$ from $P,R,T,EP,P_{log}$ 

\end{frame}

\begin{frame}
\frametitle{Result}
ICA transform among all terms.
 
$n_{independent\_components}=n_{variables}$:
\begin{equation*}
M_o=0.0015P_{Mean}-0.0074Rh+0.0005P_{Lag}+0.0349Ta_{Mean}-0.0006Eps
\end{equation*}
\begin{center}
$\rho^2=0.6787$ 
\end{center}

$n_{independent\_components}=2$:
\begin{equation*}
M_o=0.0008P_{Mean}-0.0049Rh+0.0008P_{Lag}+0.0144Ta_{Mean}-0.0001Eps
\end{equation*}
\begin{center}
$\rho^2=0.5856$
\end{center}

$n_{independent\_components}=3$:
\begin{equation*}
M_o=0.0009P_{Mean}-0.0046Rh -0.0146Ta_{Mean}+0.0008Eps
\end{equation*}
\begin{center}
$\rho^2=0.4674$
\end{center}
\end{frame}
\begin{frame}
ICA transform among same dimension terms.
$n_{independent\_components}=n_{same\_dimension\_variables}$:
\begin{equation*}
M_o=0.0015P_{Mean}-0.0074Rh+0.0005P_{Lag}+0.0349Ta_{Mean}-0.0006Eps
\end{equation*}
\begin{center}
$\rho^2=0.6787$ 
\end{center}

$n_{independent\_components}=2$: 
\begin{equation*}
M_o=0.0005P_{Mean}+0.0011Rh+0.0007P_{Lag}+0.0040Ta_{Mean}-0.0001Eps
\end{equation*}
\begin{center}
$\rho^2=0.5800$
\end{center}

We recommend the last result for an intuitive deduction that the 2 independent components reflect the water and heat control.
\end{frame}

\begin{frame}
\frametitle{Support Vector Machine}
A support vector machine constructs a hyperplane or set of hyperplanes in a high or infinite dimensional space, which can be used for classification, regression, or other tasks. Intuitively, a good separation is achieved by the hyperplane that has the largest distance to the nearest training data point of any class (so-called functional margin), since in general the larger the margin the lower the generalization error of the classifier.
 \begin{figure}[htbp]
\centering
\includegraphics[width=4cm]{Classifier.png}
\end{figure} 
\end{frame}

\begin{frame}
\frametitle{Non-linear Condition}
\begin{figure}[htbp]
\centering
\includegraphics[width=8cm]{Kernel.png}
\end{figure} 
\end{frame}
 
\begin{frame}
\frametitle{Support Vector Regression}
Minimize the sum of the empirical risk function and the structural risk function:

empirical risk function:
\begin{equation*}
\frac{1}{l}\sum_{i=1}^{l}|y_i-f(x_i)|_\epsilon
\end{equation*}
 
structural risk function:
\begin{equation*}
\frac{1}{2}w^Tw
\end{equation*}
\end{frame}

\begin{frame}
\frametitle{Example}
 
\end{frame}
 
\end{document}
