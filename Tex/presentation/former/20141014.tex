\documentclass{beamer}
\usetheme{Warsaw}
\usepackage{amsmath}
\usepackage{authblk}
\usepackage{booktabs}
\usepackage{float}
\usepackage{listings}
\begin{document}

\title{Water Balance Modelling Evaluation through Mapping the Hydrological Pattern to Information Space\\
\raggedleft Paper Script 1.0}
\date{\today}
\author {Pan Baoxiang}
\maketitle 


\begin{frame}
\frametitle{Outline}
\begin{itemize}
\item Abstract
\item Introduction
\item Theoretical Analysis of Water Balance Models
\item The Epistemic Aleatory Uncertainty Framework
\item Result
\item Discussion
\item Conclusion
\end{itemize}			
\end{frame}

\begin{frame}
\frametitle{Abstract}
\begin{itemize}
\item Motivation \& Problem Statement
\item Data \& Approach
\item Result
\item Conclusion
\end{itemize}
\end{frame}

\begin{frame}
\frametitle{Abstract: Motivation \& Problem Statement}
\begin{itemize}
\item General Picture

Conceptual Water balance models have been developed and used for climatic change impact exploration and long-range stream flow forecast.
\item Zoom in

With a similar iteration pattern but different constitutive functions and insufficient inputs, most of the existing models are capable of generating satisfying outputs over a monthly time scale, where a fuzzy gap of hydrological simulation exists between the long range water-energy correlation pattern and the detailed precipitation run-off generation mechanism. 

\end{itemize}
\end{frame}

\begin{frame}
\frametitle{Abstract: Motivation \& Problem Statement}
\begin{itemize}
\item Find the Gap

However, to smooth the interim of different time scale hydrological modelling, the legality of the iteration pattern and the constitutive functions of these models remain to be examined over the time crack.

\item Theoretical Support of The Method 
 
The Aleatory-Epistemic Uncertainty Framework 
\end{itemize}

\end{frame}

\begin{frame}
\frametitle{Abstract: Data \& Approach}
\begin{itemize}
\item Data

the experiment basins of MOPEX project
\item Approach 

Use the Aleatory-Epistemic Uncertainty Evaluation Framework to examine the information flow and source sink terms of 6 conceptual monthly water balance models over time scales from one day to a year.
\end{itemize}
\end{frame}

\begin{frame}
\frametitle{Abstract: Result}
\begin{itemize}
\item the efficiency of the constitutive functions
\item how the efficiency change with time scales
\item how is the change rate connected with the catchment characteristics
\item the state variable, its function and explanation
\end{itemize}
\end{frame}
 
\begin{frame}
\frametitle{Abstract: Conclusion}
\end{frame}

 

\begin{frame}
\frametitle{Introduction}
\begin{itemize}
\item General Picture \& Zoom In
\begin{itemize}
\item Water Balance Model
\item Evaluate Criterion
\end{itemize}
\item Organization
\end{itemize}
\end{frame}


\begin{frame}
\frametitle{General Picture}
A major realm of hydrological community is to figure out the components of the catchment water balance over different time scales.
\begin{itemize}
\item Physically-based models
\begin{itemize}
\item detailed information
\end{itemize}
\item Conceptual models and patterns
\begin{itemize}
\item Describe a coarser but valuable hydrological form
\item Available for mathematical analysis
\end{itemize}
\end{itemize}
\end{frame}
\begin{frame}
\frametitle{Water Balance Model}
\begin{itemize}
\item Budyko Curve: A Paradigm
\item Brief Introduction of Water Balance Models 
\end{itemize} 
\end{frame}

\begin{frame}
\frametitle{Evaluate Criterion}
\end{frame}

\begin{frame}
\frametitle{Organization}
\begin{itemize}
\item  Theoretical Analysis of 5 existed water balance models
\begin{itemize}
\item Disintegration
\item Constitutive Function Analysis
\end{itemize}
\item  Brief Introduction of Aleatory Epistemic Uncertainty framework \& Its Application Considerations
\item Data \& Method
\item Result
\item Discussion \& Conclusion
\end{itemize}
\end{frame}

\begin{frame}
\frametitle{Theoretical Analysis of Water Balance Models}
\begin{table}[H]
\caption{A Brief History of 5 Water Balance Models}
\begin{center}
\begin{tabular}{llc}
\toprule
Term  & Developer & Time  \\
\midrule
TMWB      & Thronthwaite \& Mather    & 1948\&1955      \\
ABCD      & Thomas     & 1981      \\
VMWB      & Vandewiele \& Xu     & 1992      \\
TPWB      & Xiong \& Guo     & 1996      \\
DMWB      & Zhang \& Potter      & 2008       \\
\bottomrule
\end{tabular}
\end{center}
\end{table}
\end{frame}

\begin{frame}
\frametitle{Basic Pattern --A Recursive Functional Equation Set}
\begin{equation*}
\left\{
\begin{aligned}
   &Structure=\left\{\begin{aligned}&Output\_Generation\\&State\_Variable\_Renewal\end{aligned}\right.\\
   &Output\_Generation (Input,State,Paramter)=Output  \\
   &State\_Variable\_Renewal (Input,State,Parameter)=New\_State\\
   &Simulation(Structure,Input,State,Parameter)\\&=\left\{\begin{aligned}&Output\\&Simulation(Structure, New\_Input,New\_State,Parameter) \end{aligned}\right.\\
\end{aligned}
\right.
\end{equation*} 
\end{frame}

\begin{frame}
\frametitle{Numerical Parts of 5 Water Balance Models}
\begin{table}[H]
\begin{center}
\begin{tabular}{lcccr}
\toprule
Term  & Input & Output & State Variable & Parameter  \\
\midrule
TMWB   & $P$,$EP$& $E$,$Q$        & $S_1$,$S_2$&$a$,$SC$        \\
ABCD   & $P$,$EP$& $E$,$Q_d$,$Q_g$& $S$,$G$    &$a$,$B$,$c$,$d$ \\
VMWB   & $P$,$EP$& $E$,$Q$        & $S$        &$a$,$b$,$c$     \\
TPWB   & $P$,$EP$& $E$,$Q$        & $S$        &$c$,$SC$        \\
DMWB   & $P$,$EP$& $E$,$Q_d$,$Q_g$&$S$,$G$     &$a$,$b$,$SC$,$d$\\
\bottomrule
\end{tabular}
\end{center}
\end{table}
\end{frame}

\begin{frame}
\frametitle{Structures of 5 Water Balance Models}
\begin{table}[H]\tiny
\begin{center}
\begin{tabular}{lll}
\toprule
Term & Output Generation Section & Variable Renewal Section\\
\midrule
TMWB
& 
$E=\left\{\begin{aligned}&EP~;P>EP\\&EP-P~;else\end{aligned}\right.$ 
& 
$S_{1new}=min(S_1+P-E,SC)$ \\
& 
$Q=\lambda\underbrace{[S_2+max(S_1+P-E-SC,0)]}_{WO}$
& $S_{2new}=(1-\lambda)WO$ \\
\\
ABCD
& 
$E=\underbrace{[\frac{P+S}{2a}-\sqrt{(\frac{P+S+b}{2a})^2-\frac{(P+S)b}{a}}]}_{EO}[1-e^{-\frac{EP}{b}}]$ 	
& 
$S_{new}=EOe^{-\frac{EP}{b}}$ \\
& 
$Q_d=(1-c)(P+S-EO)$ 	
& 
$G_{new}=\frac{G+c(P+S-EO)}{1+d}$ \\
& 
$Q_b=dG$ 	
& \\
\\
VMWB
& 
$E=min[EP(1-a^{\frac{S+P}{EP}}),S+P]$ 	
& $S_{new}=S+P-Q-E$ \\
& $Q=bS+c[P-EP(1-e^{-\frac{P}{EP}})]$ 	   \\
\\
TPWB
& $E=cEPtanh[(P+S)/EP]$ 
& $S_{new}=S+P-Q-E$ \\
& $Q=(S+P-E)tanh[(S+P-E)/SC]$ 	       \\
\\
DMWB
& $Q_d=P-\underbrace{B\_Curve(P,SC-S+EP,a)}_{W}$
& $S_{new}=W+S-R-E$ \\
& $Q_b=dG$ 	
& $G_{new}=(1-d)G+R$ \\
& $R=W+S-B\_Curve(W+S,EP+SC,b)$
\\
& $E=B\_Curve(W+S,EP,b) $  
&\\
\bottomrule
\end{tabular}
\end{center}
\end{table}
\end{frame}

\begin{frame}
\frametitle{A Systematic Perspective Toward the Constitutive Functions}
Supply-Demand Framework
\begin{itemize}
\item Extreme Boundary Conditions
\item Buckingham $\Pi$ Theorem
\end{itemize}
\end{frame}

\begin{frame}
\frametitle{A Stochastic Analytical Perspective Toward the Constitutive Functions}
We should notice when applying the water balance models at a monthly or annual  time scale, the information that inputs (here we mean precipitation and potential evapotranspiration) in a single iteration unit provides is the available water and energy in whole during that period, with the detailed hydrological processes lost in accumulation. 
We assume:
\begin{itemize}
\item Independence between local soil moisture and precipitation occurrence
\item Consistent meteorological condition during the calculating period
\end{itemize}  
\end{frame}

\begin{frame}
\begin{equation}\tiny
 \begin{split}
 \frac{\partial{f(s,t)}}{\partial t}=\frac{\partial{[\rho(s)f(s,t)]}}{\partial s}-\lambda(t)f(s,t)+\lambda(t)\int_{0}^{s} g(z,t)p_{i|z}(s-z)dz+\lambda(t)p_0(0)e^{-\lambda(t) t}p_{i|0}(s)
 \end{split}
 \end{equation}
 \begin{equation}\tiny
 p(r,t)=\lambda(t)\int_{0}^{1} f(z,t)f_{r\_depth}(r+1-z)dz
 \end{equation} 
 
 
 We are constrained by:
\begin{itemize}
\item The accumulative EP equals the observation
\item The accumulative P equals the observation
\end{itemize}
\end{frame}

\begin{frame}
\frametitle{The Epistemic  Aleatory Uncertainty Framework}
\begin{figure}[htbp]
\centering
\includegraphics[width=10cm]{3.png}
\end{figure}
\end{frame}

 

\begin{frame}
\frametitle{Considerations}
\begin{itemize}
\item Benchmark
\item Seasonal Fluctuation Inconsistency
\end{itemize}
\end{frame}

\begin{frame}
\frametitle{Method \& Data}
\begin{figure}[htbp]
\centering
\includegraphics[width=6.5cm]{flowchart.jpg}
\caption{Programming Flow Chart} \label{fig:graph}
\end{figure}
\end{frame}

\begin{frame}
\frametitle{Auto Calibration}
\begin{footnotesize}
\begin{equation*} 
\left\{
\begin{aligned}
   &Model(Parameter)=Simulation(Structure,Input,State,Parameter)\\
   &Evolution(Parameter\_Sets)=Evolved\_Parameter\_Sets \\
   &Optimization (Model,Parameter\_Sets,Observation,Evaluat\_Criterion)\\&=\left\{\begin{aligned}&Local\_best\\&Optimization(Model,Evolved\_Parameter\_Sets,Obsevation,Evaluat\_Criterion)\end{aligned}\right.
\end{aligned}
\right.
\end{equation*}
\end{footnotesize}
\end{frame}

\begin{frame}
\frametitle{"Evolution"}
Any formal system could be expressed in the language of number theory. 

The mathematical fact of the evolution in the A.I. optimization algorithms is to map the pseudo-random numbers generated by some number theory laws to the parameter space under the guide of the former generations' performance for producing some "unexpected" but useful new parameters.
\end{frame}

\begin{frame}
\frametitle{An Evolution Paradigm: PSO}
\begin{equation*}
\begin{aligned}
&Evolution(Parameter)\\=&Inertial\_Coefficient \times Parameter+\\&Self\_Experience\_Coefficient \times (Hitorical\_Best-Parameter)+\\&Social\_Experience\_Coefficient \times (Global\_Best-Parameter)
\end{aligned}
\end{equation*}
\end{frame}


\begin{frame}
\frametitle{Programming Language}
\begin{figure}[htbp]
\centering
\includegraphics[width=7.5cm]{1.png}
\caption{Development Time of Languages} \label{fig:graph}
\end{figure}
All the codes are available at the github URL: 
\begin{figure}[htbp]
\centering
\includegraphics[width=4.5cm]{2.png}
\end{figure}
\centering{https://github.com/morepenn}  
\end{frame}


\begin{frame}
\frametitle{Results}
\begin{itemize}
\item Aleatory Uncertainty
\begin{itemize}
\item Excluding State Variable
\item Including State Variable 
\end{itemize}
\item Epistemic Uncertainty
\end{itemize}
\end{frame}

\begin{frame}
\frametitle{Results: Aleatory Uncertainty\\Excluding State Variable}
The stationary point of the Aleatory Uncertainty -Time Scale Function is the exact time scale point that we could ignore the impact of iterative variable.
\end{frame}


\begin{frame}
\frametitle{Results:\\Unclustered Aleatory Uncertainty Excluding Former Inputs}
\begin{figure} 
  \centering 
  \begin{minipage}[b]{0.2\textwidth} 
    \centering 
    \includegraphics[width=1in]{unclustered3.png} 
  \end{minipage}% 
  \hspace{0.04\linewidth}% 
  \begin{minipage}[b]{0.2\textwidth} 
    \centering 
    \includegraphics[width=1in]{unclustered4.png} 
  \end{minipage}
  \hspace{0.04\linewidth}% 
  \begin{minipage}[b]{0.2\textwidth} 
    \centering 
    \includegraphics[width=1in]{unclustered5.png} 
  \end{minipage} \\[20pt] 
  \begin{minipage}[b]{0.2\textwidth} 
    \centering 
    \includegraphics[width=1in]{unclustered6.png}  
  \end{minipage}% 
  \hspace{0.04\linewidth}% 
  \begin{minipage}[b]{0.2\textwidth} 
    \centering 
    \includegraphics[width=1in]{unclustered7.png} 
  \end{minipage}% 
  \hspace{0.04\linewidth}% 
  \begin{minipage}[b]{0.2\textwidth} 
    \centering 
    \includegraphics[width=1in]{unclustered8.png} 
  \end{minipage} 
  \hspace{0.04\linewidth}% 
  \begin{minipage}[b]{0.2\textwidth} 
    \centering 
    \includegraphics[width=1in]{unclustered9.png} 
  \end{minipage}
\end{figure}
\end{frame}

\begin{frame}
\frametitle{Results:\\Clustered Aleatory Uncertainty Excluding Former Inputs}
\begin{figure} 
  \centering 
  \begin{minipage}[b]{0.2\textwidth} 
    \centering 
    \includegraphics[width=1in]{clustered3.png} 
  \end{minipage}% 
  \hspace{0.04\linewidth}% 
  \begin{minipage}[b]{0.2\textwidth} 
    \centering 
    \includegraphics[width=1in]{clustered4.png} 
  \end{minipage}
  \hspace{0.04\linewidth}% 
  \begin{minipage}[b]{0.2\textwidth} 
    \centering 
    \includegraphics[width=1in]{clustered5.png} 
  \end{minipage} \\[20pt] 
  \begin{minipage}[b]{0.2\textwidth} 
    \centering 
    \includegraphics[width=1in]{clustered6.png}  
  \end{minipage}% 
  \hspace{0.04\linewidth}% 
  \begin{minipage}[b]{0.2\textwidth} 
    \centering 
    \includegraphics[width=1in]{clustered7.png} 
  \end{minipage}% 
  \hspace{0.04\linewidth}% 
  \begin{minipage}[b]{0.2\textwidth} 
    \centering 
    \includegraphics[width=1in]{clustered8.png} 
  \end{minipage} 
  \hspace{0.04\linewidth}% 
  \begin{minipage}[b]{0.2\textwidth} 
    \centering 
    \includegraphics[width=1in]{clustered9.png} 
  \end{minipage}
\end{figure}
\end{frame}

\begin{frame}
\frametitle{Results}
when $T>365$, the influence of the state variable become insignificant in experimental basin 6.
\end{frame}

\begin{frame}
\frametitle{Discussion}
The shape of different defined Aleatory Uncertainty (Epistemic Uncertainty)- Modelling Time Scale at different basins.
\end{frame}

\begin{frame}
\frametitle{Conclusion}
\end{frame}


\begin{frame}
\frametitle{Difficulties}
\begin{itemize}
\item Calculating Speed
\item Data 
\end{itemize}
\end{frame}

\begin{frame}
\frametitle{Remaining Work \\Aleatory Uncertainty }
\begin{itemize}
\item The impact of former input in calculating Aleatory Uncertainty at different time scales
\item Did the state-variables in the models take the same effect as the former input in calculating Aleatory Uncertainty at different time scales
 \end{itemize}
\end{frame}

\begin{frame}
\frametitle{Remaining Work \\Epistemic Uncertainty }
\begin{itemize}
\item The Efficiency of the Constitutive Functions
\item The Shape of the Epistemic Uncertainty-Modelling Time Scale Curve
\item Its Relation with the Basin Characteristics
\end{itemize}
\end{frame}


\end{document}