\documentclass[10pt]{beamer}
\usetheme[]{Feather}
\usepackage[utf8]{inputenc}
\usepackage[english]{babel}
\usepackage[T1]{fontenc}
\usepackage{helvet}
\usepackage{multirow}


%-------------------------------------------------------
% DEFFINING AND REDEFINING COMMANDS
%-------------------------------------------------------

% colored hyperlinks
\newcommand{\chref}[2]{
  \href{#1}{{\usebeamercolor[bg]{Feather}#2}}
}

%-------------------------------------------------------
% INFORMATION IN THE TITLE PAGE
%-------------------------------------------------------

\title[] % [] is optional - is placed on the bottom of the sidebar on every slide
{ % is placed on the title page
      \textbf{Temporal Scale Analysis of \\
      Catchment Hydrological Models \\
      in Stochasitic Perspective}
}

\subtitle[Temporal Scale Analysis of \\
      Catchment Hydrological Models \\
      in Stochasitic Perspective]  % this is the title that appears at the bottom of the slide.
{
      \textbf{ }
}

\author[Baoxiang]
{      Baoxiang Pan  \\
      {\ttfamily baoxianp@uci.edu}
}


\date{\today}

%-------------------------------------------------------
% THE BODY OF THE PRESENTATION
%-------------------------------------------------------

\begin{document}
\renewcommand{\thefootnote}{\fnsymbol{footnote}}
\setcounter{footnote}{0}
%-------------------------------------------------------
% THE TITLEPAGE
%-------------------------------------------------------

{\1% % this is the name of the PDF file for the background
\begin{frame}[plain,noframenumbering] % the plain option removes the header from the title page, noframenumbering removes the numbering of this frame only
  \titlepage % call the title page information from above
\end{frame}}


\begin{frame}{Content}{}
\tableofcontents
\end{frame}

%-------------------------------------------------------
\section{Introduction}
%-------------------------------------------------------



\begin{frame}{Introduction}{Background} 
%-------------------------------------------------------
Hydrologic cycle takes on different patterns and requires different models at different temporal scales. 
\begin{table*} \small
\resizebox{\textwidth}{!}{
\centering
\begin{tabular}{ccc}

\hline
 \textbf{ } & \textbf{Event Scale}& \textbf{Long-Term Scale}\\
\hline
\multirow{2}{*}{\textbf{Constitutive}}  &P\footnote{Precipitation} Partitioning& \multirow{3}{1in}{P-EP\footnote{Potentail Evapotranspiration} Correlation} \\
\multirow{2}{*}{\textbf{Relation}} &ET\footnote{Evapotranspiration} Process&\\
&Q\footnote{Runoff} Routing&\\
\\
\multirow{3}{1in}{\textbf{Control Factors}} &Quantity\& Intensity of P \&EP & \multirow{2}{*}{Quantity and Timing} \\
&Soil Moisture& \multirow{2}{*}{of P \& EP} \\
&Slope,Landuse, etc..&\\
\\

\textbf{Accuracy}&$\sim$mm& $\sim$10-100mm\\
\\

\multirow{2}{*}{\textbf{Representative}}&Sacramento&\multirow{2}{*}{Runoff Coefficient}\\

\multirow{2}{*}{\textbf{Models}} &SHE&\multirow{2}{*}{Budyko}\\
            &ANN&\\

\hline
\end{tabular}
}
\end{table*}  

\end{frame}


\begin{frame}{Introduction}{Gaps--Maintain Logical Rigour} 
%-------------------------------------------------------
\begin{itemize}
\item Bridging the Scales
\begin{itemize}
\item Dimensional Analysis + Extreme Conditions 
\item Propotionality Hypothesis 
\item Maximum Entropy Production 
\end{itemize}
\item Scope \& Accuracy of Models
\begin{itemize}
\item SCS \rightarrow ABCD \rightarrow Budyko
 
\end{itemize}
 
\end{itemize} 
\end{frame}
 
\begin{frame}{Introduction}{Technique Routine} 
%-------------------------------------------------------
 
  \includegraphics[width=10.245cm]{flowchart.pdf} 
 
\end{frame}



%-------------------------------------------------------
\section{Methodology}
%-------------------------------------------------------

\iffalse
\begin{frame}{Stochastic Soil Moisture Model}{General Conception} 
%-------------------------------------------------------
\onslide<1->{
 \begin{figure*}
\centering
\includegraphics[width=11cm]{modelparadigm.png}
\caption{}
\end{figure*}
 }
\onslide<2>{
\begin{center}
 \begin{table*} \small

\centering
\begin{tabular}{ll}
\hline
\textcolor[rgb]{1,0,0}{Lisp}&\textbf{Stream}\\
\textcolor[rgb]{1,0,0}{Mathematica}&\textbf{Floldlist}\\
\hline
\end{tabular}
\end{table*} 
\end{center}
 } 
 
\end{frame}
\fi
\begin{frame}{Stochastic Soil Moisture Model}{General Conception} 
%-------------------------------------------------------
\onslide<1->
\textcolor[rgb]{1,0,0}{General Conception:}
\begin{enumerate}
\item Analyse the stochastic feature of hydrological variables given the stochastic feature of the inputs.
\item Clarify the temporal scale transfer by applying limit theory to the stochastic function.
\end{enumerate}

\onslide<2-> 
 \begin{table}[H] 
\resizebox{1.07\textwidth}{!}{
\centering
\begin{tabular}{cccccccc}
\hline 
Developer&Spatial Scale&	P&	Interception& Runoff Generation& Evapotranspiration& Leakage&	Stationary Distribution\\
\hline 
Cox \& Isham &\multirow{5}{*}{Point Scale}&\multirow{5}{*}{CPP\footnote{Compound Poisson Process}} &\multirow{5}{*}{fixed}&N&Linear&\multirow{2}{*}{N}&Gamma \\

Milly &  &&&\multirow{4}{*}{Horton}&fixed&&Exponential \\


Rodriguez && &&&Piecewise& Linear&Piecewise\\
Laios && &&&Piecewise&Exponential&Piecewise\\ 
Porporato & &&&&Linear&Linear&Gamma\\
\hline 
\end{tabular}
}
\end{table}
 
\end{frame}

\renewcommand{\thefootnote}{\fnsymbol{footnote}}
\setcounter{footnote}{0}

\begin{frame}{Stochastic Soil Moisture Model}{Conservation Function} 
%-------------------------------------------------------
 \onslide<1->
{
\begin{equation*}
\label{sbalance}
nR_{L}\frac{ds}{dt}=I(s,t)-E(s,t)-L(s,t)
\end{equation*}
}

\onslide<2->
{
\leftdownarrow{\textcolor[rgb]{1,0,0}{Differential Element Method}}
\centering{\textcolor[rgb]{1,0,0}{\Huge{$\Downarrow$}}}
\rightdownarrow{\textcolor[rgb]{1,0,0}{C-K Forward Equation}}
 
\begin{scriptsize}
\begin{equation*}
\label{basic1}
f(s,t+dt)ds\footnote{$f(s,t)ds$ denotes the probability that $s$ is within $(s,s+ds)$ at time $t$.}= \underbrace{(1-p_{rain})\Bigg \{ f(s+\Delta s,t)d(s+\Delta s) \Bigg \} }_{no-rain} +\underbrace{p_{rain} \int_{0}^{s} f(z,t)p_{i|z}(s-z+\Delta z)dzds}_{rain}
\end{equation*}
\end{scriptsize}
}

\onslide<3->
{
\leftdownarrow{\textcolor[rgb]{1,0,0}{$P_{rain}=\lambda(t)\footnote{$\lambda(t)$ denotes the average rainfall frequency.} dt$}}
\centering{\textcolor[rgb]{1,0,0}{\Huge{$\Downarrow$}}}
\underleftdownarrow{\textcolor[rgb]{1,0,0}{$dt \rightarrow 0$}}
\begin{equation*}
\label{basic3}
 \frac{\partial{f(s,t)}}{\partial t}=\frac{\partial{[\rho(s)\footnote{$\rho(s)$ denotes evapotranspiration and leakage rate.}f(s,t)]}}{\partial s}-\lambda(t)f(s,t)+\lambda(t)\int_{0}^{s} f(z,t)p_{i|z}(s-z)dz
 \end{equation*}
}
 
\end{frame}


\begin{frame}{Stochastic Soil Moisture Model}{Boundary Condition $s=0$} 
%-------------------------------------------------------
\onslide<1->
{
\begin{scriptsize}
\begin{equation*}
\label{basic00}
\begin{split}
p_0(t+dt)=&\underbrace{(1-p_{rain})[p_0(t)+\int_{0^{+}}^{\rho (0)dt} f(s,t)ds]}_{no-rain} +\underbrace{p_{rain} \int_{0}^{kdt}\int_{0}^{s} f(z,t)p_{i|z}(s-z+\Delta z)dzds}_{rain}
\end{split}
\end{equation*}
\end{scriptsize}
}
\onslide<2->
{
\centering{\textcolor[rgb]{1,0,0}{\Huge{$\Downarrow$}}}
 
 \begin{equation*}
 p_0(t)=p_0(0)e^{-\lambda(t) t}
 \end{equation*}
} 
 
\end{frame}


\begin{frame}{Stochastic Soil Moisture Model}{Boundary Condition $s=1$} 
 \onslide<1->
{
\begin{equation*}
\label{basic00}
\begin{split}
p_1(t+dt)=&\underbrace{(1-p_{rain})\times 0}_{no-rain}+\underbrace{p_{rain} \int_{1}^{1}\int_{0}^{s} f(z,t)p_{i|z}(s-z+\Delta z)dzds}_{rain}
\end{split}
\end{equation*}
}
\onslide<2->
{
\centering{\textcolor[rgb]{1,0,0}{\Huge{$\Downarrow$}}}
 
\begin{equation*}
p_1(t)=0
\end{equation*} 
}
\end{frame}




\begin{frame}{Stochastic Soil Moisture Model}{Conservation Function}
%-------------------------------------------------------
\textcolor[rgb]{1,0,0}{General Form}: 
\begin{equation*}
 \frac{\partial{f(s,t)}}{\partial t}=\frac{\partial{[\rho(s)f(s,t)]}}{\partial s}-\lambda(t)f(s,t)+\lambda(t)\int_{0}^{s} f(z,t)p_{i|z}(s-z)dz
 \end{equation*}

\textcolor[rgb]{1,0,0}{Specific Form}:
\begin{equation*}
\begin{split}
 \frac{\partial{g(s,t)}}{\partial t}=&\frac{\partial{[\rho(s)g(s,t)]}}{\partial s}-\lambda(t)g(s,t)+\\
 &\lambda(t)\int_{0}^{s} g(z,t)p_{i|z}(s-z)dz+\lambda(t)p_0(0)e^{-\lambda(t) t}p_{i|0}(s)
 \end{split}
\end{equation*}
where:
 \begin{equation*}
g(z,t)\equiv
 \begin{cases}
 0,&z=0,1;\\  f(z,t),&else
 \end{cases}
 \end{equation*}
 
 
\end{frame}

\renewcommand{\thefootnote}{\fnsymbol{footnote}}
\setcounter{footnote}{0}


\begin{frame}{Stochastic Soil Moisture Model}{Constitutive Function}
%-------------------------------------------------------
\begin{itemize}
\item Soil Moisture Replenishment  
\item Soil Moisture Loss  
\end{itemize} 
\end{frame}



\begin{frame}{Stochastic Soil Moisture Model}{Constitutive Function--Soil Moisture Replenishment}
%-------------------------------------------------------
Given:
\begin{itemize}
\item The Distribution of Rainfall Depth
\item The Inital Soil Moisture Level
\end{itemize}
Calculate:
\begin{itemize}
\item The Distribution of Infiltration Depth
\end{itemize}
 
\end{frame}

\begin{frame}{Cases in Lumped Hydrologic Models}{Point Scale Horton Runoff Generaion}
%-------------------------------------------------------
 
\begin{table}[H] 
\centering
\begin{tabular}{ccccccc}

\hline 
\multicolumn{2}{c}{Horton Runoff Generation}\\
\hline 
Constitutive 
&
\multirow{2}{*}
{
\begin{equation*}
I\vert z=
 \begin{cases}
 P&{P+z\leq 1};\\1-z &{P+z>1}
 \end{cases}
\end{equation*}
}
\\
Relation& \\
\\
P.D.F
&
\begin{equation*}
\label{point}
p_{i|z}(x)\footnote{$p_{i|z}(x)$ denotes the p.d.f. that infiltration is $x$ when the soil moisture is $z$.}=
f_P(x)\footnote{$f_P(x)$ denotes the p.d.f. that rainfall depth equals $x$.}
+\delta(x-1+z)\footnote{Dirac Function:  \begin{equation*}
 \delta(x)\equiv
 \begin{cases}
 0&x\neq0;\\\infty&x=0
 \end{cases}
 \end{equation*}  \begin{equation*}
 \int_{-\infty}^{\infty} \delta(x)dx=1
 \end{equation*}}\int_{1-z}^{\infty} f_P(u) du 
\end{equation*}
\\
\hline 
\end{tabular}
\end{table}
 
\end{frame}

\begin{frame}{Cases in Lumped Hydrologic Models}{XAJ(VIC)}
%-------------------------------------------------------
\onslide<1->
\begin{table}[H] 
\resizebox{\textwidth}{!}{
\centering
\begin{tabular}{ccccccc}
\hline 
\multicolumn{2}{c}{XAJ/VIC Runoff Generation}\\
\hline 
\\
Constitutive
&
\multirow{2}{*}
{
\begin{equation*}
I\vert z=
 \begin{cases}
 1-z-[1-\frac{P+a\footnote{Average soil moisture, $a=(1+b)[1-(1-z)^{\frac{1}{1+b}}]$}}{1+b}]^{1+b\footnote{Inhomogeneity, the larger, the more inhomogeneous.}}&{a+P\leq 1+b};\\1-z &{a+P> 1+b}
 \end{cases}
\end{equation*}
}\\
Relation&\\
\\
P.D.F.
&
\begin{equation*}
\label{xaj}
p_{i|z}(x)=f_P\bigg \{(1+b)\big [(1-z)^{\frac{1}{1+b}}-(1-z-x)^{\frac{1}{1+b}}\big ]\bigg \}+\delta(x-1+z)\int_{(1+b)(1-z)^{\frac{1}{1+b}}}^{\infty} f_P(u) du 
\end{equation*}\\
\hline 
\end{tabular}
}
\end{table}

\end{frame}
\iffalse
\begin{frame}{Cases in Lumped Hydrologic Models}{SCS Curve}
%-------------------------------------------------------
 
\begin{table}[H] 
\resizebox{\textwidth}{!}{
\centering
\begin{tabular}{ccccccc}
\hline 
\multicolumn{2}{c}{SCS Curve}\\
\hline 
\\
Constitutive
&
\multirow{2}{*}
{
\begin{equation*}
I\vert z=
 \begin{cases}
 1-z-[1-\frac{P+a}{1+b}]^{1+b}&{a+P\leq 1+b};\\1-z &{a+P> 1+b}
 \end{cases}
\end{equation*}
}\\
Relation&\\
\\
P.D.F.
&
\begin{equation*}
\label{xaj}
p_{i|z}(x)=f_P\bigg \{(1+b)\big [(1-z)^{\frac{1}{1+b}}-(1-z-x)^{\frac{1}{1+b}}\big ]\bigg \}+\delta(x-1+z)\int_{(1+b)(1-z)^{\frac{1}{1+b}}}^{\infty} f_P(u) du 
\end{equation*}\\
\hline 
\end{tabular}
}
\end{table}
 
\end{frame}
\fi
\begin{frame}{Cases in Lumped Hydrologic Models}{HBV Model}
%-------------------------------------------------------
 
\begin{table}[H] 
 
\centering
\begin{tabular}{ccccccc}
\hline 
\multicolumn{2}{c}{HBV Model}\\
\hline 
\\
Constitutive
&
\multirow{2}{*}
{
\large{
\begin{equation*}
I\vert z=P-z^{\beta}P
\end{equation*}
}
}\\
Relation&\\
\\
P.D.F.
&
\begin{equation*}
\label{xaj}
p_{i|z}(x)=f_P(\frac{x}{1-z^{\beta \footnote{$\beta$ denotes ruoff generation Coefficient, the larger, the less runoff and the more infiltration, correspondingly.}}})
\end{equation*}\\
\hline 
\end{tabular}
 
\end{table}
 
\end{frame}




\begin{frame}{Stochastic Soil Moisture Model}{Constitutive Function--Soil Moisture Loss}
%-------------------------------------------------------
Given:
\begin{itemize}
\item The Energetic Condition
\item The Inital Soil Moisture Level
\end{itemize}
We need:
\begin{itemize}
\item The Evapotranspiration and Leakage Rate.
\end{itemize} 
 
\end{frame}

\begin{frame}{Cases in Lumped Hydrologic Models}
%-------------------------------------------------------
 \begin{table}
\begin{tabular}{cccccccc}
\hline 
Linear Loss&Piecewise Linear Loss\\ 
\hline 
 \begin{equation*}
\label{linearep}
\rho (s)=EP \times s
\end{equation*}
&
 \begin{equation*}
\rho (s )=
 \begin{cases}
 \frac{\eta}{s^*} s  &s\leq s^{*}\\ 
 \eta &s^*<s\leq s_1\\
 \eta+k\frac{s-s_1}{1-s_1} &s_1<s\leq 1
 \end{cases}
 \end{equation*}\\
\hline 
\end{tabular}
\end{table}
\end{frame}

\begin{frame}{Stochastic Soil Moisture Model}{Brief Sumup}
\only<1->
{
 
\begin{scriptsize}
\begin{equation*} 
\label{ssd}
\frac{\partial{g(s,t)}}{\partial t}=\frac{\partial{[\rho(s)g(s,t)]}}{\partial s}-\lambda(t)g(s,t)+\lambda(t)\int_{0}^{s} g(z,t)f_{p}(s-z)dz+\lambda(t)p_0(0)e^{-\lambda(t) t}p_{i|0}(s)
\end{equation*}
\end{scriptsize}

\begin{tiny}
\begin{equation*}\small
\label{ssm}
\frac{\partial{g(s,t)}}{\partial t}=\frac{\partial{[\rho(s)g(s,t)]}}{\partial s}-\lambda(t)g(s,t)+\lambda(t)\int_{0}^{s} g(z,t)f_{p}\{(1+b) [(1-z)^{\frac{1}{1+b}}-(1-s)^{\frac{1}{1+b}} ] \}dz+\lambda(t)p_0(0)e^{-\lambda(t) t}p_{i|0}(s)
\end{equation*}
\end{tiny}

\begin{scriptsize}
\begin{equation*} 
\label{ssd}
\frac{\partial{g(s,t)}}{\partial t}=\frac{\partial{[\rho(s)g(s,t)]}}{\partial s}-\lambda(t)g(s,t)+\lambda(t)\int_{0}^{s} g(z,t)f_{p}(\frac{s-z}{1-z^{\beta}})dz+\lambda(t)p_0(0)e^{-\lambda(t) t}p_{i|0}(s)
\end{equation*}
\end{scriptsize}
}
\onslide<2->


As \textcolor[rgb]{1,0,0}{$b  \rightarrow 0$}, the p.d.f. of XAJ/VIC degenerates to that of Horton Runoff Generation.
 
 
As \textcolor[rgb]{1,0,0}{$\beta   \rightarrow \infty$}, the p.d.f. of HBV degenerates to that of Horton Runoff Generation.
\end{frame}


\begin{frame}{Stochastic Soil Moisture Model}{Ergodicity}
%-------------------------------------------------------
\begin{figure*}
\centering
\includegraphics[width=.75\textwidth]{s_s.png}
 
\end{figure*}
\end{frame}

\begin{frame}{Stochastic Soil Moisture Model}{Ergodicity}
%-------------------------------------------------------
\begin{figure*}
\centering
\includegraphics[width=.75\textwidth]{s_m.png}
 
\end{figure*}
\end{frame}




\begin{frame}{Stochastic Soil Moisture Model}{Ergodicity}
%-------------------------------------------------------
\begin{figure*}
\centering
\includegraphics[width=.91\textwidth]{s_sort.png}
\end{figure*}
\end{frame}


\frame{
\frametitle{Autocorrelation Coefficient}

 \only<1>
{
\begin{figure*}
\centering
\includegraphics[width=\textwidth]{day_a_corr.png} 
 
 

\end{figure*}
}
\only<2>
{
\begin{figure*}
\centering
\includegraphics[width=\textwidth]{dep_a_corr.png} 
 
 

\end{figure*}
}
\only<3>
{
\begin{figure*}
\centering
\includegraphics[width=\textwidth]{ep_r_corr.png} 
 
\end{figure*}
}
\only<4>
{
\begin{figure*}
\centering
\includegraphics[width=\textwidth]{b_corr.png} 
 
\end{figure*}
}
\only<5>
{
\begin{figure*}
\centering
\includegraphics[width=\textwidth]{initial_s_corr.png} 
\end{figure*}
}
\only<6>
{
\begin{table}[H] 
\centering
\begin{tabular}{cccccccc}
\hline 
Control Factor&$|\rho (T)|$& $\frac{d\rho (T)}{dT}$\\
\hline 
$day_a$&$+$&$-$ \\ 
$dep_a$&$-$ &$0$\\ 
$EP_r$&$-$ &$+$ \\ 
$b$&$0$ &$0$ \\ 
$S_{initial}$&$0$ &$0$ \\ 
\hline 
\end{tabular}
\end{table}

\begin{itemize}
\item $+$:positive correlation
\item $-$: negative correlation
\item $0$: not correlated
\end{itemize}
}
}

















\begin{frame}{Stochastic Soil Moisture Model}{Ergodicity --First Order Moment}
%-------------------------------------------------------
 \begin{figure*}
\centering
\includegraphics[width=.5\textwidth]{temporal_average_simulation.png}
\includegraphics[width=.5\textwidth]{temporal_average_variance.png}
\end{figure*}
\end{frame}



\begin{frame}{Stochastic Soil Moisture Model}{Ergodicity --First Order Moment}
%-------------------------------------------------------
 \begin{figure*}
\centering
\includegraphics[width=.91\textwidth]{ergodicity.png}
 
\end{figure*}
\end{frame}


\begin{frame}{Stochastic Soil Moisture Model}{Ergodicity --Spectrum Analysis}
%-------------------------------------------------------
\begin{equation*}
\label{dsbalance}
\eta R_L \times \frac{ds(t)}{dt}+E(t)=I(t)
\end{equation*} 
 
\begin{equation*}
\label{fourierch}
\eta R_L \int_{-\infty}^{\infty}e^{-i\omega t}ds(t)+\int_{-\infty}^{\infty}e^{-i\omega t}E(t)dt=\int_{-\infty}^{\infty}e^{-i\omega t}I(t)dt 
\end{equation*} 
Since:
\begin{equation*}
\eta R_L \int_{-\infty}^{\infty}e^{-i \omega t}ds(t)=\eta R_L \times i\omega s(\omega)
\end{equation*}
\begin{equation*}
E(t)=EP \times S(t)
\end{equation*}
We have: 
\begin{equation*}
\label{F}
s(\omega )=\frac{I(\omega )}{i\omega \eta R_L+EP}
\end{equation*}

\end{frame}

\begin{frame}{Stochastic Soil Moisture Model}{Ergodicity --Spectrum Analysis}
%-------------------------------------------------------
\begin{equation*}
\label{F}
s(\omega )=\frac{I(\omega )}{i\omega \eta R_L+EP_r}
\end{equation*}
Energy Spectrum Density:
\begin{equation*}
E_s (\omega)=\frac{|I(\omega)|^2}{EP_r ^2+(\omega \eta R_L) ^2}
\label{sssss}
\end{equation*}
 Is Precipitation Partition a \textcolor[rgb]{1,0,0}{White Noise Filter}? If so, 
\begin{enumerate}
\item  Larger $\omega$, $\eta$, $R_L$ lead to \textcolor[rgb]{1,0,0}{Red Noise , Auto regressive Process}. 
\item Smaller $\omega$, $\eta$, $R_L$ lead to \textcolor[rgb]{1,0,0}{White Noise, Independent Process}.
\item As $\omega \rightarrow \infty$, $E_s (\omega) \rightarrow 0$, Ergodicity Proved.
\end{enumerate}
\end{frame}
 
 
\begin{frame}{Stochastic Soil Moisture Model}{Longterm Average}
%-------------------------------------------------------
Limit $t \to \infty$, $p_0(0)=0$, $b=0$:
\begin{equation*} 
\label{main4}
EP\frac{d{[sg(s)]}}{d s}-\lambda g(s)+\lambda \int_{0}^{s} g(z)\alpha e^{-\alpha \{(1+b) [(1-z)^{\frac{1}{1+b}}-(1-s)^{\frac{1}{1+b}} ] \}}dz=0
\end{equation*}
This leads to
\begin{equation*} 
\label{jxj}
g(s)=\frac{N}{EP} \times s^{\lambda / EP -1} e^{-\alpha s}
\end{equation*}
where $N$ is constant:
\begin{equation*} 
\label{normalizationfactor}
N=\frac{EP \alpha ^{\frac{\lambda} {\ EP}}}{\gamma (\lambda /EP , \alpha)}
\end{equation*}
$\gamma(*,*)$ is Incomplete $\Gamma$ Function:
\begin{equation*}
\gamma(s,x)=\int_{0}^{x} t^{s-1}e^{-t}dt
\end{equation*}
\end{frame}
 
 \begin{frame}{Stochastic Soil Moisture Model}{Longterm Average}

 \begin{table*}  \small 
\resizebox{\textwidth}{!}{
\centering
\begin{tabular}{cccc}
\hline
$E(s)=\frac{\lambda}{\alpha EP}-\frac{\alpha^{\frac{\lambda}{EP}-1} e^{-\alpha}}{\gamma (\frac{\lambda}{EP},\alpha)}$
&
$Var(s)=\frac{\gamma (\frac{\lambda}{EP}+2,\alpha)}{\alpha ^2\gamma (\frac{\lambda}{EP},\alpha)}-E^2(s)$
\\
\hline
\begin{minipage}{.5\textwidth}\includegraphics[width=\linewidth]{average.png}\end{minipage}
&\begin{minipage}{.5\textwidth}\includegraphics[width=\linewidth]{var.png}\end{minipage}
\\
\hline
\end{tabular}
}
\end{table*}
 
\begin{table*} 
\centering
\begin{tabular}{cc}
 $\frac{EP}{\lambda}$ & Average Energy Provision \\

 $\alpha$ & Average Water Provision \\
\end{tabular}
\end{table*}
 
 
\end{frame}


\iffalse
\begin{frame}{Information Perspective}
%-------------------------------------------------------
 
 
\end{frame}


\begin{frame}{Support Vector Regression}{Tradeoff Between Bias \& Variance}
%-------------------------------------------------------
\onslide<1->
\begin{equation*}
\begin{split}
&E[h(x^{*})-y^{*}]^2\\=&E[h(x^{*})-\overline{h(x^{*})}+\overline{h(x^{*})}-\overline{y^*}+\overline{y^*}-y^{*}]^2\\
=&\underbrace{E[h(x^{*})-\overline{h(x^{*})}]^2}_{ \textcolor[rgb]{1,0,0}{Variance}}+\underbrace{[\overline{h(x^{*})}-\overline{y^*}]^2}_{ \textcolor[rgb]{1,0,0}{Bias}}   +   \underbrace{E[y^{*}-\overline{y^*}]^2}_{ \textcolor[rgb]{1,0,0}{Noise}}\\
 \end{split}
\end{equation*} 

\onslide<2->
 \begin{table}[H] 
\resizebox{1.07\textwidth}{!}{
\centering
\begin{tabular}{cccccccc}
\hline 
Regression &\multicolumn{2}{c}{\textcolor[rgb]{1,0,0}{Variance}}&\multicolumn{2}{c}{\textcolor[rgb]{1,0,0}{Bias}}&\multicolumn{2}{c}{\textcolor[rgb]{1,0,0}{Noise}}\\
Methods    & Term & Weight & Term & Weight & Term & Weight \\
\hline 
LSE& Null& 0& $E[h(x^{*})-y^*]^2$&1  & Null& 0                             \\
Ridge&$||\omega\footnote{$\omega$ is the input transform coefficient vector.}||^2$  & \lambda &   $E[h(x^{*})-y^*]^2$&1-\lambda  & Null& 0                             \\
SVR &$||\omega||^2$  & $\frac{1}{2}$ &   $max(|h(x^*)-y^*|-\epsilon,0 )$& C & Null& 0                             \\
\hline 
\end{tabular}
}
\end{table} 
\end{frame}

\begin{frame}{Support Vector Regression}{Tradeoff Between Bias \& Variance}
%-------------------------------------------------------
\onslide<1->
 \begin{table}[H] 
\resizebox{1.07\textwidth}{!}{
\centering
\begin{tabular}{cccccccc}
\hline 
Regression &\multicolumn{2}{c}{\textcolor[rgb]{1,0,0}{Variance}}&\multicolumn{2}{c}{\textcolor[rgb]{1,0,0}{Bias}}&\multicolumn{2}{c}{\textcolor[rgb]{1,0,0}{Noise}}\\
Methods    & Term & Weight & Term & Weight & Term & Weight \\
\hline 
SVR &$||\omega||^2$  & $\frac{1}{2}$ &   $max(|h(x^*)-y^*|-\epsilon,0 )$& C & Null& 0                             \\
\hline 
\end{tabular}
}
\end{table}  
\onslide<2->
\begin{equation*}
\begin{split}
 minimize \quad &\frac{1}{2}||\omega||^2+C\sum \xi\\
 s.t. \quad   \quad &\left\{
                \begin{array}{ll}
                  \Big{|}  |y-<\omega,x>-b|-\epsilon \Big{|}\leq \xi  \\
                  \xi \geq 0 \\
                \end{array}
              \right\\
 \end{split}
\end{equation*}
\onslide<3->
The \textcolor[rgb]{1,0,0}{Convex} \textcolor[rgb]{1,0,0}{Quadratic Optimization} Problem makes only \textcolor[rgb]{1,0,0}{Dot Production} between samples, thus we can apply use SVM to solve \textcolor[rgb]{1,0,0}{Non-linear Optimization} with the help of \textcolor[rgb]{1,0,0}{Kernel Trick}.

\end{frame}

%-------------------------------------------------------
\section{Field Test}
%-------------------------------------------------------

\begin{frame}{Data Set}
%-------------------------------------------------------
 
 
\end{frame}

\begin{frame}{Results}
%-------------------------------------------------------
 
 
\end{frame}

\begin{frame}{Discussion}
%-------------------------------------------------------

 
\end{frame}

\fi
%-------------------------------------------------------
\section{Concluding Remarks}
%-------------------------------------------------------

\begin{frame}{Concluding Remarks}
%-------------------------------------------------------
\begin{itemize}
\item \textcolor[rgb]{1,0,0}{ Any Lumped Hydrological Model} Has its \textcolor[rgb]{1,0,0}{Stocahstic Form} (Horton, Xinanjiang, VIC, HBV, ...), which \textcolor[rgb]{1,0,0}{Provide a Rigorous Way to make Temporal Analysis}. 
\item The \textcolor[rgb]{1,0,0}{Timing} and \textcolor[rgb]{1,0,0}{Quantity} of \textcolor[rgb]{1,0,0}{Precipitation and Evapotranspiration} determine the \textcolor[rgb]{1,0,0}{Long-term Hydrological Pattern}. The \textcolor[rgb]{1,0,0}{Accuracy of this First-Order Control is Quantized}. 
\end{itemize} 
 
\end{frame}

\begin{frame}{Take-home Message}
%-------------------------------------------------------
\begin{itemize}
\item Dynamic System Analysis of Lumped Hydrological Model.
\item Good to see Hydrologic Cycle in the Frequency Domain.
\end{itemize}  
 
\end{frame}

{\1 
\begin{frame}[plain,noframenumbering]
  \finalpage{
    \onslide<1->
  {
  \begin{figure}[H]\centering
  \includegraphics[width=0.9\textwidth]{github.png}
  \end{figure}
  \center{\textcolor[rgb]{0.5,0.5,0.5}{https://github.com/morepenn}}
  }
  \onslide<2->
  {
  \center{Thank you.}
  }
  
  }
\end{frame} 
}




\end{document}








 
