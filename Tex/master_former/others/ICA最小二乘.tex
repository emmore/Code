\documentclass{article}
\usepackage{CJKutf8}
\usepackage{amsmath}
\usepackage{listings}
\begin{document}
\begin{CJK}{UTF8}{gkai}
\title{ICA最小二乘}
\date{\today}
\author{潘宝祥}
\maketitle
  设原自变量矩阵为A,列向量分别表示降水$P_{Mean}$,径流$Rh$,前期降水$P_{Lag}$,温度$Ta_{Mean}$,潜在蒸散发$Eps$.因变量为修正后$NDVI$。最小二乘拟合结果如下:
\begin{equation*}
Mo=0.0015P_{Mean}-0.0074Rh+0.0005P_{Lag}+0.0349Ta_{Mean}-0.0006Eps
\end{equation*}
\begin{center}
$\rho^2=0.6787$ 
\end{center}

独立成分分析(ICA)基本思路为:所有自变量通过线性变换后尽可能地相互独立(非高斯性最大化)。本文使用fastICA算法,先将自变量进行独立成分分析,后应用列独立矩阵对因变量进行最小二乘拟合。最后还原为自变量A对因变量的线性拟合。
\begin{itemize}

\item 所有自变量独立成分分析后最小二乘拟合
\begin{itemize}
\item 独立变量个数等于自变量个数时,结果与直接回归相同:
\begin{equation*}
M_o=0.0015P_{Mean}-0.0074Rh+0.0005P_{Lag}+0.0349Ta_{Mean}-0.0006Eps
\end{equation*}
\begin{center}
$\rho^2=0.6787$ 
\end{center}

\item 独立变量个数为2时:
\begin{equation*}
M_o=0.0008P_{Mean}-0.0049Rh+0.0008P_{Lag}+0.0144Ta_{Mean}-0.0001Eps
\end{equation*}
\begin{center}
$\rho^2=0.5856$
\end{center}

\item 独立变量个数为3时:
\begin{equation*}
M_o=0.0009P_{Mean}-0.0046Rh -0.0146Ta_{Mean}+0.0008Eps
\end{equation*}
\begin{center}
$\rho^2=0.4674$
\end{center}
\end{itemize}


\item 仅将同量纲项(降水,前期降水,径流,潜在蒸散发)独立成分分析后最小二乘拟合(推荐,物理意义更好)
独立变量个数等于自变量个数时,结果与直接回归相同:
\begin{itemize}
\item 独立变量个数等于自变量个数时,结果与直接回归相同:
\begin{equation*}
M_o=0.0015P_{Mean}-0.0074Rh+0.0005P_{Lag}+0.0349Ta_{Mean}-0.0006Eps
\end{equation*}
\begin{center}
$\rho^2=0.6787$ 
\end{center}

\item 独立变量个数为2时(建议,因为两个独立变量体现了水,热控制):
\begin{equation*}
M_o=0.0005P_{Mean}+0.0011Rh+0.0007P_{Lag}+0.0040Ta_{Mean}-0.0001Eps
\end{equation*}
\begin{center}
$\rho^2=0.5800$
\end{center}

\end{itemize}

\end{itemize}


\end{CJK} 
\end{document}
