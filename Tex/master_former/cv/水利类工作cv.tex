% LaTeX resume using res.cls
\documentclass[margin]{res}
%\usepackage{helvetica} % uses helvetica postscript font (download helvetica.sty)
%\usepackage{newcent}   % uses new century schoolbook postscript font 
\setlength{\textwidth}{5.1in} % set width of text portion
\usepackage{float}
\usepackage{graphicx} 
\usepackage{amsmath}
\usepackage{authblk}
\usepackage{booktabs}
\usepackage{caption}
\usepackage{subfigure}
\usepackage{CJKutf8}
\usepackage{amsmath}
\usepackage{listings}
\usepackage{amsmath}
\usepackage{booktabs}
\usepackage{float}
\usepackage{graphicx} 
\usepackage{diagbox}
\usepackage{qtree}
\usepackage{tikz}
\usepackage{tikz-qtree}
\begin{document}
\begin{CJK}{UTF8}{gkai} 

% Center the name over the entire width of resume:
 \moveleft.5\hoffset\centerline{\large\bf 潘宝祥}
% Draw a horizontal line the whole width of resume:
 \moveleft\hoffset\vbox{\hrule width\resumewidth height 1pt}\smallskip
% address begins here
% Again, the address lines must be centered over entire width of resume:
 \moveleft.5\hoffset\centerline{清华大学水资所}
 \moveleft.5\hoffset\centerline{北京 100084}
 \moveleft.5\hoffset\centerline{(86) 133-6672-0253}
  \moveleft.5\hoffset\centerline{panbaoxiang@hotmail.com}



\begin{resume}
\section{教育背景} {\sl 水文与水资源工学学士,}  
                      % \sl will be bold italic in New Century Schoolbook (or
	              % any postscript font) and just slanted in
		      %	Computer Modern (default) font
                武汉大学 水利水电学院\\ 
                \\
                {\sl 水文与水资源工学硕士,}  
                      % \sl will be bold italic in New Century Schoolbook (or
	              % any postscript font) and just slanted in
		      %	Computer Modern (default) font
                清华大学 水资所\\ 
               
 
\section{计算机技能} {\sl 编程语言 \& 软件应用:} Lisp, C, Python, \LaTeX\ ; Matlab, Mathematica, R,  Mysql,  ,Git, vim, ArcGIS, Grapher, AutoCAD .\\
                {\sl 操作系统:} Linux, Windows.\\
                {Github URL:} https://github.com/morepenn
               
                
\section{数学技能}  分析,概率,统计,随机过程,数值分析,线性代数,运筹学,信息论,机器学习.
 

\section{英文成绩} 
\begin{minipage}{0.42\linewidth}
\begin{table}[H]\footnotesize
\begin{center}
\caption*{GRE}
\begin{tabular}{cccc}
\toprule
Verbal   & Quantity & Writing   &Total   \\
\midrule
152 & 170 & 3.5 & 322 \\  
\bottomrule
\end{tabular}
\end{center}
\end{table}
 
\end{minipage}
\hfill
\begin{minipage}{.64\linewidth}
\begin{table}[H]\footnotesize
\begin{center}
\caption*{TOEFL}
\begin{tabular}{ccccc}
\toprule
Reading & Listening  &  Speaking & Writing & Total \\ 
\midrule
29 & 27 & 20 & 26 & 102 \\  
\bottomrule
\end{tabular}
\end{center}
\end{table} 
 
\end{minipage}
  

                

\section{科研工作} {\sl 水文模拟}   \\
                 \begin{itemize}  \itemsep -2pt %reduce space between items
                 \item 概念性降雨产流模型: TOPMODEL, XINANJIANG Model, Shanbei Model, HyMod.
                \item  水热耦合模型随机过程分析: Monthly Water Balance Models, Budyko. Stochastic Soil Moisture Model.
                \item  最优化: Genetic Algorithm, Particle Swarm Optimizer, SCE-UA Algorithm. 
                \item 数据驱动模型: Support Vector Machine, Artificial Neutral Network.
                \end{itemize}
 
                {\sl 随机水文}  \\
                 \begin{itemize}  \itemsep -2pt %reduce space between items
                 \item 水文时间序列分析
                 \item 基于信息熵互信息的模型评价
                 
                 \end{itemize} 
                 {\sl 气象水文}  \\
                 \begin{itemize}  \itemsep -2pt %reduce space between items
                 \item 边界层理论
                 \item 蒸散发估计比较研究 
                 
                 \end{itemize} 
                 {\sl 野外实验}  \\
                 \begin{itemize}  \itemsep -2pt %reduce space between items
                 \item 黑河上游流域土壤特性采样分析与实验站建设 
                 \item 塔里木河流域植被与土水特性采样调研         
                 \end{itemize} 
                 \section{学术经历}  
 {\it 汇报\\} 
 \begin{itemize}
 \item 第二届,第三届清华大学中科院学术年会口头汇报 
 \item 2014美国地理联合会年会海报汇报
 \end{itemize}
  {\it 论文发表\\} 
  \begin{itemize}
  \item Pan, B. and Cong, Z.  Information Analysis of Watershed Hydrological Patterns across Temporal Scales (t在投). 
  \end{itemize}
 

\section{业余爱好}             

            {\it 篮球}, 院队,系队主力锋卫. \\
            {\it 二胡}, 6级.\\
            {\it 绘画}, 华中科大建筑学辅修. \\
            {\it 读书}, 科幻和科学哲学爱好者.
 

\end{resume}
\end{CJK}
\end{document}



