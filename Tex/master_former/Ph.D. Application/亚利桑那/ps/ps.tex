\documentclass{article}
\usepackage[T1]{fontenc}
\usepackage[utf8]{inputenc}
\usepackage[margin=1in]{geometry}
\usepackage{hyperref}
\hypersetup{%
unicode=true, CJKbookmarks=true, bookmarksnumbered=true,
bookmarksopen=true, bookmarksopenlevel=1, breaklinks=true,
colorlinks=false, plainpages=false, pdfpagelabels, pdfborder=0 0 0 }
\urlstyle{same}
\newcommand{\HRule}{\rule{\linewidth}{0.5mm}}
\newcommand{\Hrule}{\rule{\linewidth}{0.3mm}}

\makeatletter% since there's an at-sign (@) in the command name
\renewcommand{\@maketitle}{%
  \parindent=0pt% don't indent paragraphs in the title block
  \centering
  {\Large \bfseries\textsc{\@title}}
  \HRule\par%
  \textit{\@author \hfill \@date}
  \par
}
\makeatother% resets the meaning of the at-sign (@)

\title{Statement of Purpose}
\author{Baoxiang Pan}
\date{Ph.D. Applicant}

\begin{document}
  \maketitle% prints the titlwe block
\large{
  The unpaved muddy yard after a heavy summer rain is the heaven of earthworms, toads, plants and my childhood’s imagination. The inherent inclination to the beauty of nature mystically led me to the Department of Hydrology and Water Resources when I was choosing field to major in. The academic training during the past 6 years tells that nature will not reveal its true beauty without one's tireless work in decoding it.   The academic life is not as cosy  as I have imagined, but I hold a strong determination to continue this hard way because of the challenges and their aesthetic returns.

My undergraduate hydrology teacher used to introduce to our newbies the forefronts of the field. This very first impression of the Department of Hydrology and Water Resources in UA got concreted when I wrote my version of the SCE-UA optimizer in the following Hydrological Model class. The project seems inappreciable today as I have been accustomed to expressing academic understandings with the language of computer codes, but the joy of intellectual accomplishment  still inspires me to pursue wisdom on the academic road. 
 
As is depicted in my {\href{https://github.com/morepenn}{Github time bar}}, the probable track of my academic interests  could be roughly divided into two forks, with one focusing on hydrological data analysis and the other on hydrological simulation. The idea that well-organized data could reveal mechanism(model)  while  models could compress data in turn is widely accepted but only imprecisely described. Among all the related research works, I believe Professor Hoshin V. Gupta makes the most theoretically gracious contribution to unifying the two fields.

An important section of my graduate thesis is to apply the information-theory-based framework brought forward by Hoshin to estimate the information of hydrological observations and their connections across temporal scales. The primary results responds to the temporal upscaling scheme of stochastic process descriptions of catchment hydrological patterns as is clarified in the former chapter. I can not appreciate more for the encouragements and advices from Hoshin during this work. 

The AGU 2014 fall meeting offered me a chance to make a close look at the world geo-science community. The face-to-face communication with Hoshin, together with the 
flourishing academic atmosphere, greatly reinforced my determination to make my contribution to this field. I want to expand the hydrological information-analysis work to the frequency domain to distinguish the impacts of periodicity   in hydrological simulation. I want to make analogy and comparison between the ideas from Shannon and Kolmogorov in detecting information of systems. I want to know the connection between the information framework and the classic Bayesian framework in hydrological simulation diagnosis. I want to figure out the elastic factors that determine the long-term hydrological patterns based on catchment stochastic descriptions. I want to find out  if these ideas are too naïve or of some value through discussions with Hoshin and many other research colleagues. I want to  gain my position in the academic community. 

The intolerance of math misapplication and sloppily-formulated programming has delayed my research work from publication. It's time to put down the reservedness and hone the ideas through discussion in the community in the forefront of the field. I sincerely look forward your favourable consideration regarding my application to the Ph.D. program. 

Best wishes.
 


  
 
\end{document}
