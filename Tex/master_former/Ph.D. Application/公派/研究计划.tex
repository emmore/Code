\documentclass[a4paper, 12pt]{scrartcl}
\usepackage{amsmath,amsfonts}
\usepackage{scrpage2}
\usepackage{scrtime}
\usepackage{setspace}
\usepackage{natbib}
\usepackage{enumerate}
\usepackage{graphicx}
\usepackage{verbatim}
%\usepackage{harvard}
\usepackage{color}
\usepackage{url}

\pagestyle{scrheadings}
\setlength\headheight{25.2pt}
\ohead{\thepage}
\chead{Pan 	Baoxiang}
\ihead{Research Plan}
\ofoot{}
\cfoot{}
\ifoot{}
\setheadsepline[head]{0.3pt}
\renewcommand{\headfont}{\normalfont}
\newcommand{\sqz}{\hfill\phantom{.}}
\newcommand{\thickhline}{\noalign{\hrule height 0.8pt}}

 
\author{\today
\vspace*{1.99cm}\\
\large{CSC Funding Candidate:\hfill\phantom{.}}\\
\large{Pan Baoxiang}\\
\small Institute of Hydrology and Water Resources, Tsinghua University \hfill \\
 \\[1.99cm]
\large{Applying For:\sqz}\\
\small{Ph.D. in Department of Hydrology and Water Resources,University of Arizona}\sqz\\[-0.2cm] \small{}\sqz\\[0.3cm]
\normalsize{ }\sqz\\[1cm]
\large{ \hfill\phantom{.}}\\
\large{ }\\[2cm]
}
 
\date{}
\title{Research Plan\\ \large for CSC funding in 2015}

\begin{document}
\begin{titlepage}
\maketitle  
\thispagestyle{empty}
\end{titlepage}
\addtocounter{page}{-1}
\section{Research Background}


Professor Hoshin V. Gupta is a world famous hydrologist specifying in the philosophy, theory and practice of building, calibration and application of mathematical models, and issues on merging quantitative, fuzzy and/or qualitative data with models. He encourages students to understand and question conventional scientific wisdom, to examine underlying assumptions, to draw from ideas in other disciplines, and to think “outside the box.” His research group  has made contributions to the theory and practice of model calibration (including multi-criteria and Bayesian methods for assimilating information from data), model structure estimation, global optimization, parameter sensitivity analysis, and artificial neural networks. The group is also keenly interested in the application of emerging and futuristic technologies to hydrologic science, including for example distributed and embedded sensor networks, parallel processing computational tools, and multi-satellite sensors.

As a master candidate from Institute of Hydrology and Water Resources, Tsinghua University, I have been admiring the beautiful and meaningful work made by Hoshin for long. Through the work of my fellow apprentice Gong Wei, who studied with Hoshin in his exchange year, I learned that Hoshin is paying his attention to using items in information theory to regularized the uncertainty in the hydrological observation and simulations system. The crystalline idea greatly influenced me. As I dived into this task, Hoshin gave me some most valuable ideas. I was amazed why he could see through the system and find the 
key to to solution. He told me to use quantized entropy to represent the prior uncertainty of hydrological systems, and multi step meteorological inputs to produce current step runoff simulation, These ideas formed the soul of my first paper. 

Hoshin used to be the mathematician and modeller of the department. He holds the thinking method of a philosopher. The hydrological community is now inundated by new observation vehicles and ecological and sociological demands. We call for such insights. I believe that is why his work could have more than 20 thousand citations. I want to study with him to learn how long-lasting accomplishments  are made. 
\section{Research Expectation}
Since I have been doing research on upscaling stochastic soil moisture model and information theory applied in hydrological observation simulation evaluation, which are similar to the focus of Hoshin's research work, I would like to go along the way to reach the front of this field, specifically, I would like to make investigations in:
\begin{itemize}
\item Compare hydrological item information content and connection between time domain and frequency domain. 
\item Combine the information theoretical evaluation framework and Bayesian Uncertainty Framework.
\item Compare the ideas from Shannon and Kolmogorov in detecting information of complex systems.
\item Quantify the memory of soil moisture based on frequency analysis and stochastic soil moisture model to make temporal upcaling scheme of  the stochastic soil moisture model.
\item Figure out the dominant factors in mesoscale meteorological models and land surface models through simulation and analysis. 
\end{itemize} 

\section{Schedule}
The Ph.D. study is expected to be 3 years. My schedule is as follows:
\begin{itemize}
\item[1st year] 
\begin{itemize}
\item Course learning
\item Language Practice
\item Academic Communication
\item Paper Modification
\item Research Methods and Tools Learning
\item Wide Paper Reading
\end{itemize}
\item[2nd year] 
\begin{itemize}
\item Conform Research Area 
\item Mass Paper Reading to Deepen Understandings of The Specific Area
\item Paper Writing
\end{itemize}
\item[3rd year]
\begin{itemize}
\item Mass Paper Reading to Find Breach and Possible Research Positions
\item Graduate Thesis Writing
\item Research Position Seeking
\end{itemize}
\end{itemize}
 


\end{document}

